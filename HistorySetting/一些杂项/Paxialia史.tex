\documentclass[UTF8]{ctexbook}

\title{\textbf{Yealsor史}}
\author{无名}
\date{}
\begin{document}

\maketitle

    \pagenumbering{roman}
    \setcounter{page}{1}

    \begin{center}
        \Huge\textbf{前言}
    \end{center}~\

    Paxialia帝国的建立结束了长达百年的混乱局面,
    这是我们这个时代最伟大的成就之一。
    为了能将这份成就和荣誉分享给帝国后世的子民,
    帝王指派我们去撰写这一部史书,用以记录帝国
    的过去与将来。虽然这项任务有诸多学者参与,
    但由于史料的缺失,对于一些事件我们无法确
    保其准确性,只能提出一些猜测以供读者进行参考。
    此外,撰写史书这是一项史无前例的任务,我们都是第一
    次尝试,因此本书可能会或多或少会出现一些问题,
    这些就只能劳烦后来的读者不断去完善本书了。

    Paxialia的历史按照整理可以区分为三大时代,分别为
    传说时代、诸王时代和最后的大征服时代。传说时代
    主要对先祖流传下来的一些传说进行整理,介绍先祖们来到土地上的
    那些事情。诸王时代是国家与民族形成的那段时间,散居于这片土地
    上的先民各自建立国家。战争与灾荒充斥着整个时代,可同时,智慧
    与文明的光辉也在此绽放,雄辩家、变革家、大将军、大贤者等等
    名人层出不穷,我们主要通过各国留存下来的档案、
    人物传记、颂歌等去了解这个时代。

    虽然诸王时代是个充满了激情的时代,可是天下百姓渴望的终究是
    平稳安定的生活,而不是战火与纷争。伟大的先王Astra Kapzi
    觉察到了这点,励精图治,开启了大征服时代。对于这一时期的历史,
    我们有详尽的资料去介绍它。
\tableofcontents
\part{传说时代}
    \chapter{天鹰指路}
    根据古老的传说,先祖们曾生活在世界的彼岸,直到某天,星桥黯淡,Kelrse的大漩涡出
    现了异样,先知预言到一场巨大的灾难将会降临,可是他并不知道要如何去应付这场灾难。

不久,群鸟便带来了海洋愤怒的声音,巨大的海浪席卷大地,吞噬了诸多的生灵,将深渊中邪灵
推入人间。

鲜血,火焰,长矛,邪灵蛊惑着人,杀戮从灾难中绽放,大海冲击着人们的家园,而那些流离失
所的人,则掠夺着更多人的家园,难民们四处流窜,这远片在雪山脚下的土地也无法幸免,越来
越多的难民堆积于那,同先祖们争夺着生存的资源。

战火渐渐蔓延,灾难近在眼前,先祖们将何去何从?谁也不知道,他们想做些什么,可是什么又
是他们能做的呢?随着难民的增加,先祖们和难民的矛盾越来越多,双方都不希望爆发冲突,可
生存的压力迫使他们不得不面临这冲突。

战争似乎是不可避免的,一场很小的偷窃事件彻底引发了双方的冲突,先祖们集合在一起,气势
汹汹地前往难民们的暂居地点。当然,他们也不是待宰的羔羊,纷纷拿起随声携带或者临时制作
的武器前去对峙。

双方举起了自己的木矛,投石索也飞快地转了起来,随时准备好掷向敌人。可也就是那时,天鹰
成群,朝着雪山掠去。它们羽翼在与寒风的搏斗散落,如同下起了羽之雨,一根根羽毛就像一封
封信件,镌刻着神明的启示,向人们诉说着什么。

对峙的双方都放下了武器,一致看向了鹰群翱翔的方向。它们,正在试图去跨越那座似乎不可逾
越的白峰。

雄鹰尚且如此,更何况我们呢?在场的人都羞愧无比,宁为了一口食物而大打出手,都不愿意朝
着雪山再迈出一步,这不是懦弱是什么?只会将长矛指向自己人可算不上勇士,只有敢于迈向未
知去探索那一丝的可能性的人,才值得被冠以勇士之名。

也就这样,双方达成了和解,并且比之前的关系都缓和了不少,各自敞开心扉欢迎对方的加入,
同时也为雪山探索之事出谋划策,它们各自拿出自己携带的最好的食物,武器,衣服给那些志愿
者。

在经过很短时间的计划和安排后,数支探险队循着天鹰的落羽出发了。

    \chapter{守望者}
    群山绵延,探险者翻过了一座又一座雪山,谁也不知道,下一座的对面是什么,可能是一片
    沃土,也可能又是另一座高峰,更有可能是一片海洋,一个更恐怖的深渊。

很快厌乏了,不安和恐惧在探索的队伍中弥漫。望不到尽头的白雪,快要干瘪的粮食袋,没有谁还
有信心去翻越下一座雪山。可眼下,恰恰又一座更高的雪山等待他们去攀登。谁也不愿意去尝试,
他们也不敢再去尝试,都在胆怯害怕着什么。
paladia他们那支探险队不是最先出发的,可最先出发的那几支队伍已经早在暴风雪中失去了踪影,
留下来的的只有零星的遗骸和残留的物资。他们去了哪里?谁也不知道。我们会和他们一样吗?或
许吧,大家都是这么想的。

寒冷、死亡与荒芜是这个世界的主旋律,任何生者都不应该打破这片宁静,违反者必然会遭受来自
雪山的报复。

前进还是后退,都没有底,眼下,已然没有能够支撑他们继续走下去的食粮了,往回走或许能遇到
下一支探险队。但更令人头疼的是,白雪皑皑,风暴早已掩埋了他们来时的道路,除了点点星光与
远方的雪山,便没有了其他的指引。

眼看下一场风暴将要到来,继续呆在洞中也是不是办法。爆裂的柴火噼里啪啦地响着,没有人发言
,只是默默地感受着这所剩无几的火焰的温暖。

不能如此下去了,paladia站了起来,但其他人并没有理会他,依旧盯着燃烧着的柴火,或许心依
旧如千年的坚冰一样冰凉了吧,不会再有所触动了。paladia为了唤醒他们僵化的灵魂,从腰间将
刀抽出,毫不犹豫地将自己的左手臂斩断,扔在了火堆中。

被突然丢入火堆中的手臂吓一跳,大家惊愕地抬起了头,不解且慌张地看着paladia,同时收也不
自觉地将收放在自己腰间的刀上,大家都以为paladia可能疯了。

血液快速地被冻住了,伤口处被薄冰所覆盖,寒冷支配着他的痛觉。paladia整理好衣服,从火堆
中拾起他那被烤焦的手臂,啃了一口,随后丢给了附近的一位探险者,意思让给他们吃了。

我要去那座雪山上看看。paladia留下这句话后就拿起自己的登山杖出发了,留下了尚且还不明白
发生什么的众人。

风暴很快到来,抹去了paladia的脚印,大家认为他只是真的疯了,自杀去了,便如同忘了刚刚发
生了什么一样地继续坐在火堆旁等待救赎,只是,多了一只手臂可以啃食。

风暴中,大雪盖住了天空,白昼与黑夜不再有了分别,山洞中幽暗的火光摇曳着,将众人的影子投
射在墙壁的坚冰上,冀求山洞能够记住这群失去灵魂的可怜人。

不知过去了几天,风暴平静了,柴火中最后的火焰也熄灭了,众人平静地接受了自己的命运,整理
行装准备启程。下一支探险队没有到来,也许风暴吞噬了他们;也许白雪迷失了他们;也或许,人
们放弃了他们,不再将希望寄托在雪山之上。继续前进是他们的使命,也是他们的归宿,如何逃避
也无法规避自己的宿命,山洞中的日子无非只是拖延了一些时间让他们想清楚了这个道理而已。

可也是这时,目光中那最高的雪山之巅,突然闪烁了绚丽的光泽,就如太阳初升所绽放的光芒一般
耀眼。开始大家都很奇怪,但是一瞬间,都也立刻想到了那个疯子。众人面面相觑,本来打算各奔
东西找片属于自己永眠之地的他们都不约而同地有了新的目的地,去那座雪山看看,那里有什么,
paladia又发生了什么。

    \chapter{新世界}
\part{诸王时代}

\part{大征服时代}

\end{document}