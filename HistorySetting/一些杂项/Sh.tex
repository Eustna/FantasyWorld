\documentclass[UTF8,12pt,draft]{ctexbook}
\usepackage{amsmath}
\usepackage{amsfonts}
\usepackage{tipa}
\usepackage{graphicx}

\title{\textbf{历史}}
\author{}
\date{}
\begin{document}
\maketitle
\tableofcontents
    \chapter{简介}
            \section{Yealsor}
                Yealsor是一个地理概念,
                出自Mider-Garde文明中最古老的Tebel语言,
                意为星星落下的地方,
                后经由寻星者传遍整个大陆,
                成了西方的代称。

                Yealsor文明以河流Pukral与支流Amiso为核心,
                在交汇的Queto平原诞生了Limdis文明,
                这是整个Yealsor文明的核心。
                Queto平原以北是Slide平原,
                南方是Rmifure山脉,
                它和西边的NxRoulise山脉相连,
                将Yealsor半包围了起来。
                
                文明主要由四个民族构成,
                分别为Limdis、Aeray、Aediu以及古王朝末裔。
                他们各自主要生活在Queto平原、Pukral上游、
                Pukral东部的Drou平原。

                他们的历史可以大致分为三个时期:
                早期Limdis文明为主导的城邦政治时期;
                Lurte王国建立后所盛行的王权政治时期;
                以及Yealsor成为了区域性大国这一概念后的帝国时期。

                Yealsor地区是所有文明的源头,
                其他地区的文明最早都可以追溯到古王朝,
                传说那批走遍世界的寻星者就是古王朝的另一支末裔。
                不过可惜的是,
                Yealsor文明因为其地理的独特原因,
                始终没有成为世界的中心,
                而与南大陆的Catheliea文明一样,
                成为人们口中常谈但却又很陌生的一个文明。

                在Yealsor文明的历史中,
                经历过三次规模庞大的人口迁徙,
                第一次是气候因素导致Aeray人的北迁,
                第二次是著名的
                


                
            \section{Catheliea}
                        
            \section{Mider-Garde}
                一群自称从

            \section{Hile}

            \section{Eldue}

            \section{Krv}

            \section{Zuvaka}

            \section{X}

            \section{Jlg}
\part{文化篇}
    \chapter{信仰的力量}
        \section{早期原始信仰}
            \subsection{诸神战争}
                诸神战争是Yealsor地区十分重要的一则传说,
                经由这场战争,
                天与地撕裂,
                生和死降临,
                人类也与万物产生了区别。

                战争的胜者获得神之位格,
                化为群星在天空中永恒闪烁,
                败者则陨落大地,
                永远地被它囚禁。
                大海横亘于天际之中,
                为天上的群星监视与惩戒着大地上的万物。

                永恒漫长的时间过去了,
                一颗被群星排挤的流星陨入了大地,
                那是名为人类的星辰。

                尚未失去神格的人类在大地上建立起了古王朝,
                试图去打破囚笼,
                重新回归天际。
                海神Totr并不会容忍这种行为,
                用一场大水摧毁了整个古王朝,
                人类彻底沦为了大地之物。

                被大水冲散的古王朝的子民从此四散于各地,
                他们的信仰也因此不同,
                有的臣服于海神的力量;
                有的试图重新恢复古王朝,
                找回人类的神格;
                也有的试图去唤起所有大地之物的神格,
                共同反抗群星,
                回归属于自己的天际。
                同时,也有人在流亡途中得到了启示,
                放弃了天空,
                信仰在荒野之中。
                \subsubsection{死亡信仰}
                    在诸神战争中,
                    初始之神拥有执掌着死亡,
                    这是弑神的力量,
                    诸神忌惮它,
                    将其弥散于大地之上,
                    地上的造物也由此而会死亡。

                    但死亡不会因忌惮而消失,
                    诸神也不会因此而永恒,
                    若能掌握死亡的力量,
                    群星都将陨落,
                    这是复仇,
                    也是平等。

                    凝聚于这股弑神的力量之下,
                    人们坚信他们终将能够重建古王朝,
                    与海神展开决战,
                    斩断禁锢他们的囚笼,
                    重新回归于天际。
                \subsubsection{海神信仰}
                    古王朝的覆灭象征着人类彻底失去了神的位格,
                    但这不意味着人类彻底失去了希望。
                    洪水之后,
                    从海洋中走出几只神兽,
                    向古王朝遗址的人类传达海神的旨意:

                    \emph{  臣服吧,
                            吾乃海神Tutr,
                            向汝降下灾厄的神明。
                            如今吾将赐予汝宽恕,
                            臣服吧,
                            歌颂吾之名,
                            守护吾之名,
                            成为吾之子民
                            吾许诺会将汝送回天际,
                            与那群星同在。\\
                            \rightline{——《神语》}}

                    收到宽恕的人类背弃了他们的古王朝,
                    选择臣服于海神,
                    这样他们就能在归尘之日被大海所拥抱,
                    回归于群星之中。
                \subsubsection{万灵信仰}
                    万物之为万物是在于他们固有自在的灵,
                    这就是神的位格,
                    正如大海的灵就是海神一般。
                    万物依旧存在则必定还存在自身的灵,
                    这种灵不会随着诸神战争而被摧毁,
                    群星只是封印了灵的神力,

                    万灵信仰的人们认为这种封印只是愚蠢的催眠,
                    他们希望能唤醒万物内心那原始的野望,
                    唤醒他们自在的灵,
                    不再被群星所欺骗。

                    可随着唤灵祭司的不断呼唤,
                    人们渐渐发现,
                    大地不是囚笼,
                    而是另一片群星,
                    由万物的灵所凝聚的雄厚的完全不输群星的大地。
                \subsubsection{荒野信仰}
                    海神惩戒大地的万物,
                    无处可去的人只能前往世界的边缘,
                    那茫茫白雪的雪山。
                    然而,天鹰在此时为迷失之人指明了一条道路,
                    无畏的勇者于群山之巅为人们守望,
                    最终,他们穿越了世界,
                    抵达了荒野之上。
                    
                    荒野是另一片海洋,
                    可它不会吞噬,
                    而是用其宽阔的胸怀拥抱着一切。
                    万物在其上生机勃发,
                    但是却也暗藏灾祸。
                    荒野不是神明的使徒,
                    它是世界本来的面貌,
                    群星与诸神不过是谎言,
                    荒野才是一切的真实。
            \subsection{星空律法}
                星空是人类永恒的追求,
                对于Mider-Garde文明的人们而言,
                星空更是人生的意义所在。

                传说古老有一批寻星之人,
                攀登上了高峰,在那建起了观星塔,
                用石盘镌刻了群星的轨迹,
                把握住了世界的肌理。
                随后,这些古老的寻星者将星之声告诉人们,
                世界为何物,为什么世界会展现如此的姿态。
                命运与法则是群星的伟力,
                引导着星空之下的万物如此存在,
                循着星的轨迹,
                星空律法也就在此基础之上随之建立。

                星空是永恒不变之物,
                变换的是有着光与暗交替的大地,
                是生与死的轮回。
                这是生命的狂妄,
                亦是灾祸的根源,
                抵抗群星之人只能永远徘徊于大地之上。
                唯有信仰群星,
                遵循命运与法则,
                才能回归星空。

                可生命的狂妄最终还是击碎了群星的力量,
                律法被魔法摧毁,
                变革亦或是灭亡,
                寻星者们需要做出抉择,
                但时代并没有给他们选择的时间。
                这或许就是命运吧,
                也正是命运,
                让他们作为另一种群星,
                在历史的星空永恒闪耀着。
                \subsubsection{太阳律}
                    群星是悬于天空闪烁的星辰,
                    而最为耀眼的那颗莫过于带来光亮的太阳。
                    光芒为大地带来了温暖,
                    万物在这太阳的律法之下得以庇护。

                    金黄的天际线横亘于苍穹,
                    一批迷失在星海中的寻星者们得到了太阳的启示,
                    伟大的力量展现在世人的面前,
                    那抹去一切黑暗的光芒,
                    是主宰一切的力量。

                    火焰与光芒是太阳律的主题,
                    它的子民精于铸造和殿宇,
                    创造了诸多不朽的传奇。
                    神庙林立,
                    是太阳的辉煌,
                    而太阳的骑士,
                    让这辉煌照耀了更遥远的蛮荒。
                \subsubsection{双月律}
                    光明中潜藏黑暗,
                    却在黑暗中绽放光明,
                    双月诠释了轮回与转换的关系,
                    正如生与死的轮回。

                    万物无法避免死亡的降临,
                    可死亡也无法阻挡新生,
                    比太阳更为宏大的力量控制着世间的一切变化,
                    纵使光芒万丈的太阳,
                    也要经历晨昏的交替,
                    在夜晚陷入睡眠。

                    双月掌管了生与死,
                    这是万物的始与终,
                    也是命运之线的两极。
                    侍奉生死的寻星者们认为太阳并非永恒,
                    因此选择了那同样悬于天空的双月作为引导之星,
                    建立了双月律法,
                    把握了万物的成长规律,
                    在农学与医学领域颇有建树。
                \subsubsection{星辰律}
                    最古老的寻星者们并没有皈依太阳与双月,
                    他们依然不断地前往群山之巅,
                    仰望并记录着永恒的群星,
                    星辰浩如烟海,
                    穷尽每一颗星星的轨迹是几乎不可能实现的事情,
                    更何况时有星辰陨落,
                    也常有新星诞生。

                    但对于星空的追求是这群古老的寻星者的永恒信仰,
                    遵循星辰律法的他们笃信,
                    哪怕这是一项不可能完成的事业,
                    他们也要尽力去实现,
                    因为拥抱群星就是拥抱永恒。

                    星辰律是最古老的律法,
                    但也是最不为人知的律法,
                    它的门槛很高,
                    需要大量的观星经验和数学能力,
                    而如此高的门槛也使得他们能在数学以及天文上有更多的成就。
            \subsection{圣树信仰}
                大陆的中北部是一片巨大的森林,
                在其中有着一颗传奇的树木,
                那被人们称为始祖大树。
                它的根系十分广阔,
                汲取着大地丰饶的精华,
                诸多大树甚至从它的根部长出,
                依赖着这颗始祖大树而生存。

                始祖大树十分巨大,
                这要求了它必须十分坚硬,
                这份硬度甚至可以比肩青铜。
                Eldue的人们居住在此,
                环绕着大树建立了自己的城市,
                利用始祖大树发展了自己的文明。

                生命是大树的主题,
                可恰恰生命是错综复杂的体现,
                Eldue人难以适应这份复杂,
                最终引燃了一场大火。
                无数生灵哀嚎,
                熊熊的火光点亮了夜空,
                数年的大火让一切化为了灰烬,
                那颗始祖大树也位列其中,
                成为了Eldue人永恒的传说。

                生命是如此奇妙,
                灾难后的Eldue人并没有因此而崩溃,
                反而捧起了始祖大树的灰烬,
                前往了更远的地方,
                巨木森林比以往更广阔也更丰饶。
                \subsubsection{巨木森林}
                Eldue的人们认为,
                这颗始祖大树是他们的守护灵,
                也是这片巨木森林的守护灵。

                森林是他们宝贵的生活经验,
                丰饶的物产能填饱他们的肚子,
                各式各样的草药也治愈了诸多疾病和伤痕。
                而最重要的是,
                森林的精灵一直在聆听着他们的诉求,
                每一个Eldue人都能在其中得到共鸣,
                那是始祖大树发出的呼唤,
                这片森林欢迎他们的到来。

                得到了森林的馈赠,
                人们也得表达自己的敬意,
                巨木信仰也由此产生,
                人们折下始祖大树的树枝,
                种植在更远的地方,
                扩大这片巨木森林的范围。
                \subsubsection{轮回大树}
                当Eldue人彻底将自己与森林融为一体后,
                他们认为生命是大树的一种形式,
                自己也不例外,
                死后融入土壤,
                为大树提供营养,
                而灵魂则会沿着根系,
                进入大树体内,
                永恒地和其中的精灵一起守护这片森林。

                所以在出生的时候,
                他们的父母会前往始祖大树并折下一枝树枝,
                种在孩子成长的土地上,
                以方便未来能顺利地找到始祖大树的根系。

                生命的形态是多样的,
                并非是所有的树枝都能成长为茁壮的树木,
                那些腐烂的残树对于人们而言是不祥的象征,
                这在他们眼中,
                意味着无法进入大树,
                是被大树排斥的存在。
                
                人们会歧视他们,
                甚至会将他们驱逐出这片巨木森林,
                这为那场焚毁一切的大火埋下了灾厄的火种。
                \subsubsection{小圣树}
                大火后的巨木森林化为了一片灰烬,
                人们并不伤心,
                因为森林焚烧时流亡的岁月已经让他们的眼泪枯竭了,
                能够得以返回故土反而是欣慰的,
                哪怕只是一片灰烬。

                大火是炽热的,
                却让Eldue人冷静了下来,
                在流亡与战争中,
                他们不断思考生命的意义,
                腐烂的残树未必是不祥的征兆,
                森林允许着一切生物的生老病死,
                容忍着一切腐烂和病变等残破的生命形态,
                土壤会将一切吸收,
                变为一种平等的姿态。

                回到灰烬之地的居民们有的在上面重新建立起了家园,
                有的为了向生命赎罪,
                抓起一撮始祖大树的灰烬,
                前往了更远的地方,
                用灰烬中的始祖精华,
                栽培出了一棵又一棵的小圣树,
                在巨木森林外又培育了一片圣树森林。
            \subsection{众神之王}
            \subsection{山灵信仰}
                \subsubsection{巨石崇拜}
                \subsubsection{主神教}
                \subsubsection{上神教}
                \subsubsection{群星崇拜}
            \subsection{其他信仰}
                \subsubsection{母神崇拜}
                \subsubsection{光辉神殿}
                \subsubsection{回归之路}
        \section{神话与传说}
            \subsection{古王朝神话体系}
            \subsection{Catheliea神话}
            \subsection{圣树神话}
            \subsection{南Hile神话}
            \subsection{法王圣典}
            \subsection{西大陆神话体系}
                \subsubsection{无尽海洋与三大放逐}
        \section{宗教的兴起}
            \subsection{海神教}
            \subsection{Catheliea多神教}
            \subsection{星空律法}
            \subsection{日神教}
    \chapter{不同文化的社会关系}
        \section{氏族关系}
            不同的文化会有自己相应的个体的身份标识,
            名是最基本的标识,
            而根据名的位置大致可以区分为两类,
            即名在前与名在后这两类。
            在名的基础上,
            会有些许附加,
            比如基于血缘关系的有姓氏和家族,
            基于地缘关系的有城邦、国家和地区,
            也有基于社会关系的职业、身份地位等。
            在比较特殊的地区也会有用学院作为附加标识的,
            由于一些宗教和王室原因,
            还有会加上灵、位和格作为附加标识。

            也有一些比较特殊的形式,
            比如职业不是作为附加标识而是成为主要的标识取代了名,
            或者出于文化习俗关系有需要变名以及人物具有称号等等。
            大体上有姓氏-名、名-家族、地区-职业-名和名-地区这几种形式,
            不过具体上根据社会地位、历史要素和文化习俗等关系,
            会在这几种姓氏上有些许拓展。

            后为了方便讨论,
            可以提前对于具有同一意义的附加标识进行合并,
            比如基于血缘关系的有姓氏和家族,
            二者都可以用姓来表达,
            当然,有些时候有家族内部有不同姓氏的存在,
            在那个时候再单独说明,
            在一般意义上,
            姓氏与家族可以等价看待。
            在地缘关系上,
            包括比较特殊的学院,
            都可以视为区,
            而其他由于比较特殊,
            根据需要单独说明。
            \subsection{Yealsor文明}
                \subsubsection{Limdis}
                    Limdis文明的构成形式比较复杂,
                    有君主制的城邦,
                    也有采取贵族元老院制度的城邦,
                    还有的是民主制城邦。
                    所以他们具体而言称呼的形式不是很相同,
                    但主要还是采取基于\emph{名-区}的形式,
                    一些有势力的贵族为了标识自己的身份会在其后加上自己的家族。

                    家族对于贵族派的城邦而言也是贵族身份的象征,
                    一些城邦有针对贵族的特权,
                    后来随着社会的发展,
                    一些家族名也就成了贵族的爵位名称。
                \subsubsection{Aeray}
                    Aeray人不是一个血缘概念,
                    也不是严格上的地区概念,
                    它是Limdis对于野蛮人的称呼,
                    指的是早期周围那些比较分散的小部落。
                    这些部落有些是基于血缘而形成,
                    也有些是地区氏族,
                    来源于其他部落的流民、出逃的奴隶以及其他因素居住于此的人。
                    他们和Limdis文明类似,
                    使用的也是\emph{名-区}这种形式,
                    对于由血缘关系构成的氏族部落,
                    他们的则是\emph{名-姓}的形式,
                    二者在结构上没有大的区别,
                    只是附加标识的来源有所差别。

                    但随着Lurte城邦的建立,
                    越来越多的Aeray抛弃了他们的姓氏,
                    选择使用Lurte作为他们的附加标识,
                    不过随着人口的增加,
                    他们又恢复了姓氏的使用,
                    形成了\emph{名-区-姓}的形式。
                \subsubsection{古王朝}
                    古王朝的称谓主要采用的是\emph{名-姓}的形式,
                    虽然后来覆灭了,
                    他们的后裔基本上也是采取这种形式。
                    不过在海神教地区,
                    有些海神祭司会在自己的名前加上\emph{Kuara}表示自己的身份,
                    意思为海的使徒,
                    表示为\emph{Kuara 名-姓}。
                    这是海神教里面位的一种,
                    其下还有\emph{Pito},
                    意为海的臣民,
                    为海神教信徒的位。
                    还有一种位是\emph{Bestra}
                    在古王朝中意为回归天空,
                    这种位会由海神祭司在归星仪式中赐予给死亡的海神教信徒。
                    与之相对的有一种称之为\emph{Wuli}的位,
                    这是腐朽之人的意思,
                    会降临在忤逆海神之人的身上,
                    久而久之也成为了海神教信徒对于非信徒的蔑称。
            \subsection{Catheliea文明}
            \subsection{学院文明}
            \subsection{Hile地区}
            \subsection{山民城邦}
        \section{社会结构}
    \chapter{度量衡与纪年法}
        \section{公制单位}
            \subsection{早期Hile文明的单位制}
            \subsection{单位制的融合与统一}
            \subsection{书库的建立与度量衡的规范}
        \section{古王朝单位制}
            \subsection{古王朝的遗产}
            \subsection{遗迹的勘测与测量}
        \section{学院单位制}
            \subsection{星体的观测}
            \subsection{公制单位的原始模板}
            \subsection{星制与月制}
        \section{其他单位制}
            1
        \section{书库纪年法}
            \subsection{Hile历法}
                1
            \subsection{公制历法}
                公制历法也是广义上的书库纪年法,
                是书库建立后对世界各个文明的历史进行编撰时发明出来的纪年制度,
                随着公制单位一起成为后世最普遍采用的纪年法。
                \subsubsection{公历元年}
                    公制历法选取以造成古王朝覆灭的那场大水为纪元元年,
                    并且选择用季来主要表示,
                    将一年拆为四个季,
                    每个季有三十个月,
                    每个月以暗月到亮月的交换为界,
                    其中包含了有三十天。
                \subsubsection{四季}
                    四个季节分别为春夏秋和冬,
                    根据气候变化的特征命名,
                    一个季有三十个月,
                    即有九百天。
                    可以根据具体季数除去四所得的余数确定记载的季数处于什么季节。

                    在其他纪年法中四季分别有不同的月份表示,
                    不过按照公制历法省去了这些特殊的表示,
                    统一用数字进行表达。



        \section{学院历法}
        \section{四大纪元}
        \section{王朝历法}
    \chapter{习俗与仪式}
        \section{Yealsor}
            \subsection{城邦同盟的文化习俗}
                \subsubsection{成年礼}
            \subsection{后古王朝的文化习俗}
                \subsubsection{归星仪式}
                \subsubsection{海祭}
                \subsubsection{礼拜日}
                    歌颂海神的名
                \subsubsection{唤灵仪式}
                \subsubsection{抚剑典礼}
        \section{学院文明观星仪式}
    \chapter{职业与官职的演变}
    \chapter{凝固历史的建筑}
    	\section{Yealsor文明的建筑特色}
            Yealsor文明的建筑总体有两类风格,
            一类是Limdis文明的城邦文化所诞生的建筑,
            而另一类是古王朝末裔
    \chapter{语言和发音}
        \section{Mider-Garden的通用语言}
            根据古老的寻星者的研究,
            他们为自己所说的话语总结了一套规则,
            并为此编撰了一本《语言学》。
            在其中,
            他们定义了一些符号作为音节,
            语言也就此诞生,
            这里简单介绍下其语言的内容。

            首先他们规定了元音有7类,
            分别为

            

\part{政治篇}   
    \chapter{Yealsor文明}

        Yealsor文明是Yider大陆四大文明之一,
        它位于NxRoulise山脉东方,
        处于大陆的中央。
        东方是一片大草原,
        南面则是Finua高原,
        北临大海,
        地理位置较为孤立,
        与其他文明的交互在早期并不多。

        Yealsor文明由两条主要的分支构成,
        分别为Erosa山脉北部Silder平原的Orddnisy文明和南部Queto平原的Limdis文明。

        两类文明早期分别独立地发展,
        后因战乱导致了融合,
        Yealsor文明也从此诞生,
        它继承了Limdis文明的文化、思想、艺术和语言文字等,
        在Orddnisy文明上更多的是继承了其宗教信仰和技术、工程、自然和魔法等知识。

        在两类文明之中的还有其独立发展的一些小分支也隶属于Yealsor文明,
        比如早期因为战乱而南下脱离了Limdis文明的Aeray人,
        他们在山地丘陵地区和当期居民一起共同建立了自己的Aeray文明,
        但在Lurte帝国时期,
        随着帝国的扩张和征服,
        他们归顺并回归了Limdis文明,
        放弃了自己的语言和制度,
        只保留了一些独特的传统文化和习俗。

        位于东部的Aediu文明也是如此,
        他们最早并不属于Limdis文明,
        但随着Limdis文明的发展和扩大,
        这些游牧部落效仿它建立了自己的城邦,
        学习和使用Limdis文明的语言文字并建立了相似的制度。
        久而久之他们也被同化为了Limdis文明中的一员,
        虽然更远方的Aediu人依然保持着自己的游牧传统和早期的文明习惯,
        但始终没有离开对Limdis文明的依附,
        没有能够独立出来成为一个自主的文明体系,
        因此他们也是Yealsor文明边缘的一部分。

        \section{Yealsor文明历史概述}

            对于Yealsor文明而言,
            史学界一般认为可以区分为五个时代,
            分别为早期的城邦时代
            中期的帝国时代和战国时代,
            以及中后期的宗教时代,
            或者也有人认为中后期的应该是殖民时代,
            因为宗教作为一个历史现象,
            其跨度涵括了帝国时代中后期乃至到殖民时代结束,
            但并不具有时代特色,
            宗教最巅峰的时候是宗教战争,
            可那时候的时代特色是Yealsor文明被殖民统治的事实,
            而并非被殖民统治下爆发出的一场宗教战争。

            最早期时,
            Limdis文明是以氏族部落的形式组织起来的,
            北方的Orddnisy文明则稍微有些不同,
            作为寻星者的后裔之一,
            他们是一种贤者主导的合作部落,
            并不以血缘为纽带,
            而是通过敬仰同一圣人或圣物而实现部落成员之间的认同和协作。

            随着人口的增长和农业的发展,
            具有知识和威望的贤者们掌握了更多的权力和财富,
            他们对人、地区、语言和知识进行分级,
            改造他们原始信仰为宗教,
            建立了城邦和伟大的神迹,
            Orddnisy文明率先进入了奴隶制社会。
            
            南方的Limdis文明也是类似,
            随着分工的细化,
            部落的长老们成为了管理者,
            大量财富向他们汇集,
            他们建立了城邦来保护他们的财产,
            成为了Limdis文明的第一批奴隶主。
            由于农业规模的扩大,
            他们以Namid湖为中心,
            向四周扩张,
            征服了周围的部落,
            将那里的部落民变为了奴隶,
            或者赶走了他们。

            城邦时代主要以Limdis文明为主,
            城邦之间征伐不断,
            大大小小的城邦不断兴起、发展和毁灭,
            在一场场冲突和妥协中,
            民主制和君主制应运而生。
            民主制下的民主同盟破坏了Limdis文明的旧秩序,
            这导致了Pashi战争的爆发,
            不过在这场战争后
            Zipor城邦夺回了曾经霸主的地位,
            也重新奠定了整个Limdis的秩序,
            维护了Limdis文明的平稳和和平。

            
            但种脆弱的秩序随着Aeray人的入侵而崩溃,
            城邦之间放下了互相的恩怨,
            共同组成了城邦同盟,
            君主制城邦Xiuna在此时影响力增长迅速,
            和民主同盟一起向Zipor城邦和诸多共和制城邦发起了挑战。
            Aeray人的威胁解除后,
            这种挑战很快转变为了裁决战争,
            战争后,
            君主制与民主制城邦彻底与共和制的城邦决裂,
            前者享受战争的胜利,
            后者则是互相之间加强合作,
            积极寻求变革。

            而此时Lurte城邦在Limdis文明边缘位置开始兴起,
            它吸收了城邦同盟时期流亡的Aeray人,
            也借鉴了Xiuna城邦君主制统治模式。
            但对于Lurte城邦而言,
            最大的问题还是附庸关系,
            繁盛的Lurte城邦不再满足于臣服于Calom城邦,
            Lukadilous战争爆发。

            这场战争虽然是Lurte城邦的独立战争,
            但后面变成了民主制、君主制和共和制的再一次战争。
            战争的结果虽然是Lurte城邦的独立,
            可Xiuna城邦和民主同盟葬送了自己的未来,
            天平再次向Zipor等共和制城邦倾斜。

            战败后的Xiuna城邦从此一蹶不振,
            之后不久也因继承问题而产生了分裂,
            Lurte城邦帮助二公主夺得了王位,
            获得了Xiuna城邦的继承权。
            但Zipor城邦不会放任如此,
            否则Lurte将成为一个无法撼动的强大存在。
            第二次Lukadilous战争爆发,
            Lurte城邦展现出了其惊人的军事能力,
            它们攻陷了Ziopr城邦,
            获得了战争的完全胜利,
            它也从此成为了Lurte王国。

            Limdis文明在Lurte王国出现后进入了历史的新进程,
            王国面对突然扩大的领土,
            进行了诸多的制度的改变,
            建立了封建制度的雏形。
            虽然此时Lurte依然是一个小王国,
            可整个Limdis文明已经进入了帝国时代,
            一个伟大一统的大帝国在这个时代诞生,
            为Yealsor文明奠定了最坚实的基础。

            王国飞速扩张,
            原来的共和制城邦被拆为了的封建领,
            元老院贵族们离开城邦,
            在乡野建立了自己的永久领地。
            君主和贵族之间的矛盾化解,
            而紧接而来的就是王国和民主同盟的冲突。
            但这结果是毫无悬念的,
            城邦被攻陷,
            市民会议被民政院取缔,
            王国统一了Queto平原。

            统一让王国的野心膨胀,
            它发达了数场南征,
            征服了诸多Aeray文明的城邦和部落,
            确保了王国南方的安宁。
            此时虽然名义上依然是Lurte王国,
            但在人们的共识中,
            这个王国依然是个庞大的帝国。
            帝国随后又发动了数场东征去征服东方大草原的游牧民族,
            可常年的征战已让帝国疲惫,
            征战并不顺利。
            扩张的停滞也让国内的矛盾愈发尖锐,
            暴乱、腐败和反叛成为帝国内部很严重的问题。

            一场错误的放逐和一场失败的征战让帝国走入了末路,
            旧日的民主同盟爆发了叛乱,
            贵族们分食着帝国,
            帝国王室无法阻碍帝国的分裂。
            贵族领主们袖手旁观,
            帝国精锐的战士们被民主同盟的市民们屠戮殆尽。



            





            % Yealsor文明主要有四个民族,
            % Limdis、Aeray、Aediu和古王朝末裔,
            % 早期Limdis、Aeray人都围绕着Queto平原居住,
            % Aediu人则生活在Pukral的东侧Drou平原。
            % 古王朝末裔居住在古王朝的遗址上,
            % 也就是Pukral入海口处的三角洲平原Slide。

            % 约公历100季的时候,
            % Queto平原的Limdis人从氏族公社开始转向奴隶制,
            % 扩张和掠夺赶走了Aeray人,
            % 逼迫他们不得不朝南迁徙,
            % 前往Pukral的上游。

            % 此时的Aeray其实和Limdis在文化上并没有多少区别,
            % 或者在Limdis赶走Aeray前二者称之为一个民族也并没有多少问题,
            % Limdis和Aeray的主要区别在于Limdis主要还是以血缘氏族为主,
            % 而Aeray人则没有主要的氏族,
            % 主要由各个氏族被流放的人以及各种原因来到此地居住的人构成。

            % 对于这段时期,
            % 古王朝末裔有相关史料对其记载了,
            % 但古王朝末裔的历史相较于Queto平原较为独立,
            % 我们需要单独说明。
            % 而Limdis和Aeray这边,
            % 没有具体的史料,
            % 我们只能根据Aeray人的神话和传说进行一些猜测,
            % 以还原这段时期的历史。
            
            
        \section{城邦时代}

            Limdis文明于LC1200左右能进入青铜时期,
            那时候也产生了一批极具影响力的大型城邦,
            例如Zipor、Avalon和Briarwod等。
            他们早期是氏族的联盟,
            后面基于氏族长老开会的形式建立了元老院制度,
            共和制也因此出现。
            在元老院的主持下,
            氏族之间化解了各自的矛盾,
            重新分配了土地和奴隶,
            原来氏族社会被城邦中的市民社会所取代,
            人们通过阶级而进行分工。
            而在LC1154,Zipor的长老会议上,
            贵族、市民、乡民、流民和奴隶等阶级第一次以制度化的形式在Yealsor文明中出现,
            不同阶级的人被以不同的方式加以区别和对待,
            这也标志着Limdis文明正式进入了城邦时代。


            城邦时代主要以Limdis文明为主,
            早期的城邦是共和体制,
            城邦由元老院和元老会议主导,
            但在不同地区的城邦具体实际情况中,
            元老院的权力平衡很容易被打破,
            城邦常常被僭主所控制。
            虽然僭主的权力并不会延续太久,
            不过随着城邦的发展,
            人民和元老院之间的矛盾愈发地尖锐,
            这导致民本思想在社会广泛传播,
            君主制和民主制就在这种思潮下诞生了。

            在民主制城邦独立为民主同盟时,
            它们和当时Limdis最强盛的Ziopr城邦爆发了战争,
            虽然战争以民主同盟的胜利而结束,
            可这也为后续的Pashi战争埋下了冲突。

            Pashi战争后,
            Zipor成为了主要的执剑人
            在Zipor城邦的干涉下,
            整个Limdis文明减少了许多不必要的战争,
            诸城邦在Zipor的干涉下选择了妥协。



            Limdis由于不断地掳掠奴隶,
            规模逐渐扩大,
            慢慢地在Queto地区建成了大大小小的城邦。
            当时氏族贵族依旧掌握有很大的权力,
            早期的城邦多数还是以贵族政治为主。

            后面一些不满于贵族压迫的平民揭竿而起,
            推翻了氏族贵族的统治,
            建立了民主的共和制度,
            而有些则是开明的君主制度。
            整个Limdis文明分成了两股势力,
            代表平民的君主与民主制,
            以及代表旧氏族贵族的元老院制度。

            双方表面上井水不犯河水,
            各自相安无事,
            但是背后却是暗潮涌动,
            都时刻提防着对方。

            而就在僵持的时候,
            也就是公历324季,
            Pashi城邦遭到了Aeray人的入侵,
            这迫使两派不得不放下互相之间的恩怨,
            共同组建联军以应对来自Aeray人的威胁。

            可Aeray人的威胁一旦解除,
            同盟很快就分崩离析,
            再次分为了两派,
            可这次对峙的局面并没有持续多久,
            一场奴隶暴动点燃了双方之间的冲突。
            爆发了裁决战争。

            战争的结果是君民派取得了胜利,
            虽然是惨烈的胜利,
            不过他们在后面的很长一段时间里获得了众多城邦之间的主导权。
            而这段时间也是整个Limdis文明最辉煌的时候。
            期间,Limdis文明在艺术、文化、建筑与魔法等领域取得了举世瞩目的成就,
            这成就了Limdis文明,
            也是整个Yealsor文明的基石。

            君民派取得了主动权,
            但这并不代表贵族派一方就彻底退出了舞台,
            他们改革军事制度,
            革新武器装备,
            积极拓展城邦之外的土地,
            Lurte城邦在这个背景下被建立了起来。

            按照Lurte城邦的传统,
            他们需要有个王来统治他们,
            可这个触动了贵族们的根本利益,
            请求被无情地拒绝,
            请愿被残暴地镇压,
            Lurte城邦和贵族派之间爆发了战争,
            君民派随后介入,
            这就是公历375季的Lukadilous战争。

            贵族派一方知道自己不能两方同时作战,
            于是很快地便同意了Lurte城邦的诉求,
            转而专心对付君民派的联军。
            军事改革的成果在这场战争中出现了,
            再加上贵族派中出现了一位天才般的伟大将军,
            战事频频告捷,
            贵族派重新控制了Limdis文明。

            此后,Limdis文明不断走向衰落,
            君民派不断地被分化,
            诸多君主也不再站在市民一方,
            反而是求助于当地的氏族贵族,
            而平民们一方面要赔付战争的税款,
            另一方面还要为贵族们提供享乐,
            生活苦不堪言,
            直到第二次Lukadilous战争。

            Lurte是Aeray人建立的,
            虽然被外王所统治,
            但从Lukadilous战争独立开始,
            Lurte城邦就离开了Limdis文明的控制,
            成为了Aeray人的王国。
            不过王室毕竟是Limdis人的后代,
            这意味着Lurte城邦虽然离开了Limdis文明的政治圈,
            但文化上依然沿袭着Limdis文明的传统。
            随着越来越多的Aeray人的加入以及王室的通婚,
            双方在各方面开始融合。

            Lurte独立的消息很快便传开了,
            Aeray人闻讯而来,
            远方的Aeray氏族也不远万里来宣誓效忠,
            Lurte城邦邦很快成为了Lurte王国,
            拥有众多的人口和封臣。

            Lukadilous战争后,
            随着奴隶制进一步的发展,
            许多市民城邦形成了垄断集团,
            那些垄断集团和本地贵族相互勾结,
            慢慢演化成了寡头城邦。
            同时君主制的城邦也一步步在加强中央集权,
            渐渐和市民城邦分道扬镳,
            君民派不久名存实亡。

            公历422季,
            Sadrce城邦新兴的寡头奴隶主带兵强行解散了公民大会,
            这意味着重要的市民城邦Sadrce城邦也倒向了寡头制,
            使得本就岌岌可危的民主制雪上加霜。
            为了保护民主制,
            保护市民的权利,
            民主派组成了联军,
            共同讨伐Sadrce城邦那些解散公民大会的奴隶主。

            贵族派城邦不可能作壁上观,
            当即派出援军,
            护民战争爆发,
            与此同时,
            君主制的城邦知道贵族派在解决了平民派后下一个目标就是他们,
            虽然战争早期都还是摇摆不定,
            但随着Elamin城邦的沦陷,
            君王派正式参战。

            当然,
            参与战争的君主们也是各怀鬼胎,
            集中的权力使得他们已经不能再满足于城邦制度,
            吞并和征服的愿景在他们心中埋下了种子,
            Lurte王国的君王也不例外,
            他利用联统的身份将Lurte王国也拖入了这场战争,
            护民战争也由此演化为了第二次Lukadilous战争。

            Aeray人的介入让这场战争彻底失衡,
            那位伟大的野心家\emph{Alicyt-Lurte-Dimne},
            Lurte王国的君王,
            同时也是Aoupne城邦的君主,
            结束了这一切的混乱。
            从此,
            Limdis文明步入了王权时代。
            \subsection{僭主和民主}
                Xiuna城邦是Euro地区的一座城邦,
                它兴起于公历211季,
                通过征服和扩张收编了很多部落,
                是Queto地区最古老的一批城邦之一。

                Xiuna早期是基于当地氏族贵族的贵族共和制,
                随着Xiuna城邦的扩张,
                大量的公民和奴隶加入了Xiuna,
                随着扩张而来的还有新的氏族部落,
                他们作为新的当地贵族也加入了元老院,
                但始终难以融入其中。

                公历227纪,
                漫长的荒季到来,
                饥荒蔓延了整个Xiuna城邦,
                平民和贵族之间产生了激烈的冲突,
                大大小小抢粮事件接连发生,
                元老院紧急召回了最近的军团,
                这支军团隶属于Farane家族管辖,
                \emph{Ielous-Xiuna-Farane}因此也顺利带着军队进入了Xiuna城邦。

                但\emph{Ielous}显然并不想帮元老院平叛,
                虽然Xiuna城邦承认了他们家族的权利,
                但归顺的屈辱一直笼罩在Farane家族之上,
                \emph{Ielous}并不例外,
                在进入了Xiuna城邦之后便一路狂奔,
                包围了神庙和贵族府邸,
                强行解散了元老院,
                最后在市民的欢呼中开放了贵族们的粮仓,
                并且宣布自己从此统治Xiuna城邦。

                Farane的军团进入Xiuna后,
                释放了城邦外的贵族,
                因为他们也是被征服的部落,
                且他们也在城外拥有军队,
                所以\emph{Ielous}不打算为难他们。
                而元老院所属的公民兵已不再支持贵族长老们,
                所以这场政变很快就结束了,
                随后元老院贵族们被驱逐,
                Xiuna城邦迎来了王政时代。

                Xiuna政变成功的消息马上传遍了各地,
                此时也受饥荒影响的Firom、Sadrce、Fudge和Lapuse等城邦随后驱逐了他们的元老院,
                不同于Xiuna城邦,
                他们的起义并没有特定的领导人,
                是公民兵和市民自发的行为,
                在赶走贵族长老们后,
                除了Firom城邦选择效仿Xiuna城邦实施选举君主制外,
                其他三个城邦基于元老院扩大了议会人数,
                形成各具特色的民主制度。

                Sadrce城邦的民主基于他们原本的官僚制度,
                选举官僚成为他们的民主特色,
                Fudge的民主则是体现了早期代议制民主的特点,
                他们采取基于财产关系的选举制度,
                不同于其他城邦,
                妇女儿童和奴隶虽然不具备被选举资格但也被赋予了选举权,
                他们能够通过自己的力量选举自己的请愿官,
                在元老院争取自己的权利。
                Lapuse的民主不同于前面二者,
                他们选择采取了直接民主,
                元老院由公民大会和公审广场完全代替,
                成为整个Queto地区最民主的地方。
            \subsection{城邦同盟}
                公历200季到300季时,
                Limdis在Queto平原上建立了大大小小的城邦,
                诞生了璀璨繁荣的Limdis文明。
                Queto平原有五个主要地区,
                位于河岸西侧有Euro、Hulin和Viruan地区,
                这里是Limdis文明的核心地区,
                多数大型城邦集中在这三个地区,
                而河岸东侧和南部的分别是Godue和Aptone地区,
                那里的城邦主要以种植和放牧为主,
                因此城邦规模并不大,
                聚落也分部地很分散。
                
                公历324季初,
                Aeray不断骚扰Godue地区,
                为对抗Aeray人的劫掠,
                Godue地区的小城邦组成了Godue同盟,
                这是Queto城邦同盟的雏形。

                随着越来越多的Aeray人的到来,
                Godue同盟很难再维持对于Aeray人的清剿。
                河岸东部是Queto地区重要的粮仓,
                Aeray人的骚扰自然也影响到了河岸西侧,
                但此时那些城邦的王公贵族们显然不会在意这些微不足道的价格上涨。

                但这对于一般的市民是毁灭性的打击,
                很多人在这个时候沦为了奴隶,
                为了保护自己的公民,
                民主制城邦组成了赎奴同盟,
                双方约定遇到了同盟内的公民要尽力去将他们赎回。
                同时,
                Aptone地区的城邦为了自保也组成了Aptone同盟,
                之后很快和Godue同盟合并为了河东城邦同盟。

                让那些王公贵族们真正感受到危机的是Pashi城邦的沦陷,
                城邦被洗劫一空,
                连通河岸两侧的交通枢纽被Aeray人占领,
                河岸西侧的食物来源被截断,
                饥荒随之蔓延了整个Queto地区。

                越来越多的公民沦为了奴隶,
                奴隶暴动频发,
                公民兵也随之不断减少,
                整个Queto地区的城邦都意识到了,
                再不采取行为,
                等待他们的就是灭亡,
                尤其是王公贵族们,
                频繁的动乱已经严重影响到了统治。

                君民派的Xiuna城邦、Firom城邦和Sadrce城邦在公历324季15月组成了军事同盟,
                随后与河东城邦同盟合并为Queto城邦同盟,
                贵族派的城邦不久后也加入了城邦同盟。
                当月,
                同盟组成了共计56k的联军收复了Pashi城邦,
                恢复了河岸东西两侧的交通。

                为了彻底铲除威胁,
                Xiuna城邦的君主\emph{Kusa-Xiuna-Farane}带着22k的联军深入东部的Godue地区,
                于公历324季25月18日在Uptou平原大捷,
                击溃了Aeray人的主力部队,
                Aeray人被赶到了Queto地区的边缘地带,
                从此再无大规模进攻的能力。
                
                在赶走了外族势力后,
                城邦同盟有试图在政治和经济领域加强合作,
                然而基于城邦利益关系,
                以及君民派和贵族派的纠纷,
                最终因为裁决战争而分崩离析。

                战争将整个城邦同盟区分重新分为了公民同盟和传统同盟,
                将君民派和贵族派的对抗直接展现了出来。
                Lukadilous战争后,
                公民同盟被强制解散,
                整个Queto地区渐渐地又弱化了城邦同盟的概念,
                且随着寡头奴隶主的兴起,
                君民派和贵族派也都发生了分裂和整合,
                各个城邦以结盟的方式组成了一个又一个小同盟,
                城邦同盟几乎已经瓦解。
            \subsection{裁决战争}
                Uptou战役的胜利让\emph{Kusa}声名大噪,
                君民派在整个Queto地区的声望高涨。
                为了追求城邦同盟的主导权,
                他安排其子\emph{Prset-Xiuna-Farane}出游城邦同盟的各个城邦以增加见识与威望。

                君民派的得势很快就影响到了贵族派城邦,
                保障公民权利,
                建立公民会议等诉求层出不止,
                但贵族派的元老院自然不会同意这种请求,
                狠狠镇压了那些试图改革的市民以及小贵族。
                内部的改革无法实现,
                改革派们试图去寻求外部的力量,
                于公历325季2月4日,
                在Lemsa公民兵\emph{Tuber-Lemsa}的策划下,
                组织了一场谋杀,
                而谋杀对象正是拜访Lemsa的\emph{Prset}。

                这场谋杀引爆了Xiuna和Lemsa之间的战争,
                \emph{Kusa}虽然很痛惜自己儿子的遇害,
                但这对于他而言是一个机会,
                如果能取得这场战争的胜利,
                君民派的影响力将会被扩大。

                贵族派不会眼睁睁地看着自己的势力被君民派的夺走,
                立即采取了行动,
                组织了联军准备前往Lemsa给\emph{Kusa}施压。
                君民派同时也自发地组织起联军支援Xiuna城邦。

                双方剑拔弩张,
                但是却不敢爆发冲突,
                依然希望能通过某种更温和的方式去解决这一问题。

                公历325季6月24日,
                Lemsa被\emph{Kusa}攻陷,
                双方联军依然在Lemsa不远处继续观望着。

                当季7月1日,
                各方在Lemsa城邦内举行城邦大会,
                共同商讨Lemsa的处置问题。
                双方起初就谋杀事件展开了激烈的争论,
                君民派认为是Lemsa城邦的元老院没有尽到保护贵宾的义务,
                而贵族派一方反复强调\emph{Prset}的死应该怪罪于刺客本身,
                并不应该怪罪于元老院。

                双方都理解对方的用意,
                但为了自己的利益只能各执一词,
                始终没有谈妥。
                直到Zipor城邦的代表提出了对于\emph{Kusa}的质问,
                认为这一切谋杀都是被君民派所策划的,
                用自己儿子的生命去换取自己的野心。

                这一番言论直击了君民派的心思,
                打破了争论的平衡,
                而\emph{Kusa}恼羞成怒,
                当即下令逮捕了Zipor城邦的代表,
                战争爆发了。

                深入腹地的贵族派联军当即撤军,
                原本用来施压的军队没法直接进行作战,
                需要撤回休整后才能作战,
                但在撤军途中,
                由于统军\emph{Gulaz-Kalins-Mortie}的误判,
                联军在Rudis谷地遭受了伏击,
                几个大型城邦的主力部队被歼灭,
                战争的天平直接倾向了君民派一方。

                贵族派临时快速征召和雇佣了一些士兵,
                重新整合撤回的联军,
                稍作休整后在Santa平原与君民派的联军进行了决战。
                君民派以微弱的优势惨胜了这场会战,
                但双方都元气大伤,
                没法继续作战。

                公历325季16月14日,
                双方签订了停战协议,
                Zipor等贵族派城邦承认了\emph{Kusa}对于Lemsa城邦的统治,
                这场战争宣告结束。
            \subsection{辉煌时代}
                君民派取得了裁决战争的胜利后,
                成为了Queto地区的主导。
                城邦同盟被拆为了公民同盟和传统同盟,
                公民同盟内部的君王派和市民派开始走向了不同的方向,
                传统同盟之间则是进一步地加强合作,
                积极寻求变革。

                君王派最主要的城邦是Xiuna,
                赢得了裁决战争后,
                Lemsa被\emph{Kusa}收入囊中。
                因为难以同时管理两座城邦,
                所以\emph{Kusa}在两座城邦中间的一座山丘上建立了一座恢弘的卫城,
                王宫从Xiuna城邦搬入Kusanor卫城,
                Xiuna和Lemsa交由当地民政官管理。

                \emph{Kusa}看似是将自己的权力下方到了民政官,
                但是城邦的税赋依然会送抵到卫城,
                王室控制着城邦的财富,
                城邦的公民兵也渐渐地被王室所属的奴隶近卫取代,
                财富和奴隶一点点地通过城邦抵达卫城。

                为了彰显自己的伟大功绩,
                \emph{Kusa}在卫城中建造了一座巨大的柱林殿作为君王的神殿,
                同时召集了整个Queto地区有名的土魔法师为他铸造了一座巨大的石像立于神殿前方的广场。

                平民派这边在战争之后,
                为了彰显自己的民主制度,
                他们修筑了大量的公共建筑,
                最为有名的是Fudge城邦,
                利用叠柱的手法建造了Vrmau浴场,
                并通过引水桥在浴场中形成了小瀑布。

                Queto地区的第一部成文法也在此时诞生,
                誉为民主之星的Lapuse城邦编撰了他们的第一部公民法,
                规定了刑罚、民事纠纷、兵役制度以及税赋等内容,
                裁判所这一新机构诞生,
                取代了以前的公审制度。
                这种制度很快便传遍了整个公民同盟,
                旧的公审广场被改造为开放剧院,
                为市民提供了更多的娱乐活动,
                戏剧艺术在这个背景下大放光彩。

                繁荣笼罩着整个公民同盟,
                大量的开支花费在建筑、艺术等方面,
                他们在军事上的开支仅仅只是维持在战争前的水平,
                反观传统同盟,
                那里积极地扩充军备,
                同时进行了相应的军事改革,
                最出名的是Zipor的Cinuoa改革,
                这项改革主要由三点,
                一是许诺奴隶退役后的公民身份和金钱以及土地,
                从城邦中征集有意愿从军的奴隶,
                建立了早期以奴隶为主的职业军人制度;
                二是收缴小贵族和富农的土地作为城邦的财富,
                以扩充城邦的军费支出;
                三是军事租赁制度,
                允许公民出资并抵押财产以租赁城邦的奴隶兵外出扩张土地,
                获得的报酬按照5:4:1的比例分给城邦、个人以及奴隶兵。

                辉煌时代是Limdis文明最繁荣的时代,
                民主的光辉在此绽放,
                君主的荣耀也于这个时代展现,
                而贵族们不屈不挠的精神,
                成就了黄昏的辉煌。
                在这个时代,
                一颗新星悄然升起,
                它吸收了黄昏的光芒,
                以释放初升的晨光,
                这颗新星就是Lurte。
            \subsection{Lurte城邦}
                Calom城邦是贵族派的主要城邦之一,
                在经历了裁决战争的战败后,
                他们也积极地寻求改革,
                最先需要处理的是土地问题。

                不同于Zipor的改革,
                Calom城邦并没有通过没收小贵族的土地以作为城邦土地,
                他们的Taylonia改革主要侧重点在于税赋制度,
                具体改革内容有两点,
                一是免除贵族的土地税,
                不过贵族需要负责公民兵的训练和征募,
                当地贵族对征募的公民兵拥有指挥权,
                但需要服从城邦的征召。
                二是承认公民持有土地并允许公民向城邦购买土地,
                公民持有土地需要缴纳土地税,
                而土地税的税额为其产出的十分之一。

                Lurte城邦也就在这种背景下建立了,
                当时在Calom城邦有一名没落贵族名为\emph{Firwust-Calom}
                但此时他已经被剥夺了贵族头衔,
                只是一般的商人。

                贵族的荣耀一直藏在他的心中,
                他利用自己多年来从商获得财富和那早已被剥夺的贵族头衔,
                在Calom城郊招募了一支军队,
                但很快他的计策败露,
                他和他的那支不大的军队被当地贵族追杀,
                沦为了流寇。

                后来\emph{Firwust}在当土匪的时候遇到了同为土匪的Aeray人,
                双方一拍即合,
                策划并洗劫了追杀他们的那位贵族,
                那位贵族和他的家人也死于屠刀之下。

                土地被城邦收回,
                \emph{Firwust}此时以高价买下来这片土地,
                从此他们就有了据点,
                Aeray人的其他土匪们纷纷来到他的土地上寻求庇护,
                这批土匪在拥有了土地之后便慢慢转变为了农民,
                \emph{Firwust}所索求的那支军队也就如此莫名其妙地由Aeray人偷偷组织起来了,

                渐渐地,
                越来越多的Aeray人来到\emph{Firwust}的土地,
                虽然针对城邦方面宣称他们为奴隶,
                但如此下去很难再瞒过城邦,
                因此\emph{Firwust}决定带着他的军队前往更远的地方。
                \emph{Firwust}不知道的是,
                他的事迹早已传遍了Aeray人的各个部落,
                被他们当作神话中的王看待,
                因此当他抵达Kureta地区时候,
                当地的人们热烈地欢迎他。

                \emph{Firwust}也就顺着他们在此处修建了据点,
                称之为Lurte。
                随着越来越多的Aeray人的到来,
                Lurte很快从一座据点成长为了一座村庄,
                \emph{Firwust}将他从商多年所学得的知识全部教授给了当地的Aeray人,
                他们开垦山林,
                修筑城墙,
                种植农田,
                短短三季便发展成了一座小型的城邦。

                Lurte的动静Calom城邦不可能坐视不管,
                为了避免冲突,
                \emph{Firwust}将他的事情向Calom的元老院汇报了,
                因为Lurte当时具有了一定的武装力量,
                Calom城邦也不敢轻举妄动,
                不过\emph{Firwust}的表现很配合,
                愿意向城邦提供税赋,
                因此元老院也就妥协了,
                虽然被Aeray人占据着,
                但把Lurte视为一种属邦看待,
                对城邦也不是一件坏事。

                得到了Calom城邦元老院的承认后,
                \emph{Firwust}被Lurte城邦的居民推举为王,
                城邦快速地扩大,
                Lurte的居民积极地学习Limdis文明的文化,
                大量的学者前往拜访Queto最辉煌的Xiuna城邦。

                等到\emph{Firwust}晚年时,
                Lurte已经发展成为一座颇具规模,
                能和Calom相互抗衡的城邦了。
                Lurte的崛起带动了Calom的繁荣,
                Calom似乎是沉醉在这繁华之中,
                丝毫没有注意到其根据已经摇摇欲坠。
            \subsection{Lukadilous战争}
                公历344季到公历375季,
                Lurte城邦用30季左右的时间在Queto地区崛起成一个新兴城邦,
                而此时\emph{Firwust}也已从以为意气风发的小伙子成为了一名白发苍苍的老人,
                此时的他已近百岁。
                最终于公历375季1月4日与世长辞,
                在风雪中随着他的荣耀一同下葬。

                长子\emph{Valine-Lurte-Anumis}是合法的继承人,
                理应继承\emph{Firwust}的王位,
                但\emph{Valine}不满足于继续向Calom缴纳税赋,
                希望Calom城邦的元老院能够承认他的王位,
                承认他作为Lurte的君王是一名贵族,
                可以免除土地税,
                以军事援助作为补偿。

                这触动了Calom的根本利益,
                元老院不可能同意这种事情,
                \emph{Valine}因此只能去Calom用武力请愿,
                而元老院误解了\emph{Valine}的用意,
                认为他试图进攻Calom城邦,
                于是半路伏击那支请愿的队伍,
                \emph{Valine}在树林中被乱箭射死。

                兄长突然去世,
                次子\emph{Durand-Lurte-Anumis}接替了王位,
                这场请愿自然地演化为了战争,
                虽然Lurte城邦早已准备好了战争的借口,
                但是兄长的离世刚好促成了这场战争,
                群情激愤下,
                \emph{Durand}亲自率领军队进攻Calom城邦,
                此时为公历375季3月7日。

                这是Lurte城邦建立以来的第一场战争,
                他们广泛学习各个城邦的军事理论,
                而\emph{Durand}儿时就随着使者出使过Xiuna城邦,
                并在晚年的\emph{Kusa}身旁陪伴了几年。
                作为\emph{Kusa}的学生,
                他熟练掌握了骑兵部队的指挥方法,
                利用机动性阻断了支援Calom的贵族们的军队。

                Taylonia改革没有获得理想的效果,
                分散在各地的士兵没有办法有效地汇合,
                Calom孤立无援,
                只能看着城墙外不远处的攻城塔一点点地建造起来。

                围城期间,
                Calom向Zipor求援,
                Zipor知道此时组织援军已经来不及了,
                但出于盟约的考量,
                同时也妄想能顺势获得Calom和Lurte的控制,
                Zipor还是联合贵族派组织了联军支援Calom。

                此时,
                Zipor不知道的是,
                远方的另一个强大的城邦正在蠢蠢欲动。
                Xiuna早就得知了Lurte与Calom开战的消息,
                也预测到了Zipor会介入这场战争,
                Cinuoa改革带来的社会问题已经不可能通过Zipor内部的高压手段解决了,
                而此时的这场战争是一场机遇,
                如果能获得Lurte和Calom,
                Zipor能缓解改革军制带来的经济压力,
                更多的土地能够被许诺给士兵们,
                Lurte和Calom的财富也能源源不断地涌入Zipor,
                使其成为整个Queto地区最强大的城邦。

                Xiuna不会坐视不管,
                Zipor大军出动会导致城邦的防守薄弱,
                他们是意识到了这点,
                但在巨大的利益面前,
                妄想麻木了元老院贵族们的判断,
                将防御交给了那些不堪一击的小城邦盟友。
                Xiuna为了不打草惊蛇,
                也没有告知其他城邦他们的计划,
                直到贵族派的联军已经远去,
                他们才突然出现在Zipor的边境上。
                
                Xiuna围困了Zipor,
                而此时Zipor正在和Lurte鏖战,
                一时难以脱身,
                为了尽早地救援都城,
                Zipor放弃了自己在Calom获得的巨大优势,
                和Lurte讲和,
                让Calom承认了Lurte的独立,
                Lurte也相应地退出战争,
                Lukadilous战争进入了第二阶段。

                Zipor的围攻比Xiuna想象中的艰难,
                围困的天数也早已超出了Xiuna的预期,
                但反抗的力量丝毫不减。
                眼看着贵族派的援军越来越近,
                深入腹地的Xiuna本就在围攻中损失了大量的士兵,
                更难以对付后续抵达的Zipor援军,
                因此很快主动撤离了。

                战争并没有因此结束,
                在Xiuna的唆使下,
                Firom、Sadrce和Fudge等也对Zipor宣战,
                其他平民派城邦也紧随其后参与了战争,
                整个Queto地区再次进入了战火之中,

                宴席终将结束,
                Xiuna的狂妄葬送了他们的野心,
                腐败侵蚀了民主的骨髓。
                Zipor的解围让Xiuna丧失了战争的主动权,
                双方在Wiling地区僵持,
                互有输赢,
                直到Zipor的一位年轻的贵族主动请缨,
                

            \subsection{黄昏时代}
                Lukadilous战争后,
                君民派失去了他们的势力,

            \subsection{第二次Lukadilous战争}
                Lurte城邦的君主和Xiuna王室通婚,
                拥有法统继承权。

        \section{王权的集中与妥协}
                \subsubsection{放逐时代}
            \subsection{古王朝的后裔}
                \subsubsection{海神教}
                \subsubsection{Bestra}
                \subsubsection{Atorous战争}
                \subsubsection{南迁新都}

        \section{Yealsor帝国}
            \subsection{双王邦盟}
            \subsection{大开化}
            \subsection{Lurte驱逐}
    \chapter{大陆南部的世界}
    \chapter{学院政治}
    \chapter{Hile半岛}
        \section{古Hile文明}
        古Hile是Hile半岛南部地区以及Inhelong海峡沿岸的总称,
        它历经五个时代,
        为传穗时代、英雄时代、Hile化时代、阋墙时代和城邦联盟时代。

        最早是公历100季左右,
        由Hile半岛南部Milian森林中的Roulise文明主导,
        当时的Roulise文明掌握了灵麦的种植方式,
        所以他们的农业特别繁荣,
        吸引了很多周围的部落前来学习,
        因此这个时代也称为传穗时代。

        但繁荣的Roulise文明挡不住野蛮的掠夺,
        北方高地的流民洗劫了他们的城邦,
        Roulise文明的人们被迫流亡,
        或沦为了高地人的奴隶,
        废墟上重新建立城邦的他们早已失去了往日的辉煌,
        只能在诸多崛起的城邦中苟且求生。

        大约公历220季,
        北民骚乱,
        古Hile进入了英雄时代。
        许多伟大的人物诞生于这个时代,
        诗人们将他们的故事传唱,
        编成了历史上最早的文学作品《英雄史诗》。

        英雄们成为帝王,
        带领他们城邦的人民获得了大量的财富和奴隶,
        这因此也让这些他们被贪婪女神盯上,
        荣华和金钱腐蚀他们的灵魂,
        将他们拉入了深渊。
        《英雄史诗》用一则又一则的神话讲述了Noxine、Duling、Gromun等城邦的兴衰,
        英雄们都如同宿命一般,
        不约而同地进入了贪婪的深渊。

        《英雄史诗》虽然是一部文学作品,
        但它是研究Hile地区英雄时代的唯一史料,
        当时Hile南部的城邦遭受了来自山地城邦入侵,
        双方有战争也有融合,
        整片地区陷入一阵混乱,
        很难留下具体的史料,
        只能通过当时时代的人们口口相传,
        也就成了这部《英雄史诗》。

        英雄时代的混乱持续了百季,
        公历337季,
        Alas城邦的大帝\emph{Krahim-nia}在Hile半岛南部的Greaft平原建立起了秩序,
        这位伟大的帝王将一生致力于结束城邦分立的状态,
        他克制住了自己的欲望,
        利用多种手段征服了平原地区的城邦。

        虽然高地城邦依然没有纳入统治,
        但他们无不表示臣服,
        英雄时代在这位征服了贪婪女神的英雄手中结束。
        南Hile的文化在英雄时代已渗透进了高地城邦,
        但在\emph{Krahim-nia}统治时期,
        文化跨过Aliki山脉向着更北部的Hile半岛扩散,
        这个时候称之为Hile化时代,
        奠定了今后整个Hile地区的文化基调。

        %总共五个区域,北部平原,中部山脉,南部高地,南部平原,西部沿海

        但随着大帝\emph{Krahim-nia}的逝去,
        Alas城邦很快便失去了它的影响力。
        公历342季,
        Alpue战争爆发了,
        Aliki山脉的城邦背叛了他们对于Alas
        过去许下的承诺,
        并没有派出援军,
        虽然Alas最终击退了Prvlaxi帝国的进攻,
        可Greaft平原与Aliki山脉分裂的时代也到来了。
        
        Alas与Aliki山脉诸多城邦的战争持续了数十季,
        此时在海峡另一岸,
        Prvalaxi帝国的新帝王养精蓄锐,
        蓄势待发,
        公历411季再度入侵了Alas。
        第二次Alpue战争最终以Alas灭亡告终,
        古Hile进入了最后的时代:城邦联盟时代。

        最终,经历过大大小小的诸多次战争后,Aliki山脉诸邦纳入Prvalaxi帝国的领土,
        古Hile文明宣告灭亡。
        \section{后Hile时代}

        \section{宗教战争}


    \chapter{大山的居民}
    \chapter{极西世界的文明}



\part{经济篇}
    \chapter{古老城邦政治的商业贸易}

    \chapter{农作物}

    \chapter{阶级的意识}

\part{艺术篇}

\end{document}