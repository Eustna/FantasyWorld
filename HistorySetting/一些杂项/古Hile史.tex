\documentclass[UTF8,12pt]{ctexbook}
\usepackage{graphicx}
\title{\textbf{Hile史}}
\author{}
\date{}
\begin{document}
\maketitle
\tableofcontents
\chapter{简介}
    \section{古Hile文明简介}
        古Hile(希尔)是Hile半岛南部地区以及Inhelong海峡沿岸的总称,
        它历经五个时代。最早是前文明纪700年左右,由Hile
        半岛南部的Rose文明主导。

        大约在文明纪前200年,北民骚乱,古Hile进入了英雄时代。
        许多伟大的人物诞生于这个时代,诗人们将他们的故事传唱,
        编成了人类历史上最早的文学作品《英雄史诗》。

        随后的文明纪100年左右,Alas的大帝在Hile半岛南部的Greaft平原建立起了秩序,
        以Alas为核心的南Hile文化向北部Hile半岛扩散,这个时候也称之为希尔化时代,
        奠定了今后整个古Hile地区的文化基调。

        可是,随着Alas的首领Krahim-nia的逝去,Alas很快便失去了它的影响力。
        文明纪225年,奥普战争爆发了,Aliki山脉的城邦背叛了他们对于Alas
        过去许下的承诺,并没有派出援军,虽然Alas最终击退了Prvlaxi帝国的进攻,
        可Greaft平原与Aliki山脉分裂的时代也到来了。
        
        Alas与Aliki山脉诸多城邦的战争持续了百余年,于此时,在海峡另一岸,
        Prvalaxi帝国的新帝王养精蓄锐,蓄势待发,于文明纪361年再度入侵了Alas。
        第二次奥普战争最终以Alas灭亡告终,古Hile进入了最后的时代:城邦联盟时代。

        最终,经历过大大小小的诸多次战争后,Aliki山脉诸邦纳入Prvalaxi帝国的领土,
        古Hile文明宣告灭亡。
    \section{后Hile时代简介}
        随着最后一座城邦Rano归顺Prvalaxi帝国后,古Hile文明宣告覆灭,从此进入了后Hile时代。
        虽然Prvalaxi帝国并不能严格意义上属于Hile文化,但是在后续统治中被Hile文化所渗透,
        广义上也是可以视为Hile文明的一员,而这一过程称为泛希尔化时代。

        Prvalaxi帝国的统治有持续三百年,可三百年里,帝国内部的矛盾越积越多,最终,一场宫廷政变
        使得整个Prvalaxi帝国在一夜之间崩溃。Hile地区的督管在宫廷政变中被杀害,随后Hile地区群龙无首,
        Hile地区的分的各自为政,建立了大大小小的公国,战国时代也由此进入。

        战国时代早期,Hile地区的公国还是名义上属于Prvalaxi帝国,可随着Prvalaxi帝国实力的不断削弱,
        各个公国也不满于再被Prvalaxi的统治,纷纷独立,成立自己的王国,战国时代进入了顶峰。
        
        300年的战乱使得Hile地区苦不堪言,任何有雄心的帝王都迫切地期待去结束这一局面。
        最终于文明纪956年,Xirudia王国的第八位国王Vita-IIX完成了一统Hile的伟业,建立了第一王朝。

        第一王朝的第三位帝王Vita-X极富野心,他渴望于Aliki山脉北方的领土,因而发动了北征。
        Vita大帝一共发动了五次北征,最终在仰望Hile半岛北部的Andlass山时坠马身亡。

        随着Vita大帝的陨落,Xirudia帝国也随之分崩离析,各地农民不满于高额的战争税赋,发动了起义。
        而Xirudia帝国内部大大小小的诸侯以及城邦纷纷独立,Hile地区再度进入了小战国时代。

        Vita大帝没有名义上的子嗣,但有个私生子Avalon,他并非无能之辈,利用自己的身份在战乱中重建了
        Xirudia帝国,称为第二王朝。

        Hile半岛中部那些城邦伙同Aliki山脉的城邦共同签订了同盟条约以应对第二王朝的威胁,从此,Hile分成了两块势力
        :第二王朝与城邦同盟。
    \section{宗教战争时代简介}
        第二王朝和城邦同盟之间的斗争持续了有百年之久,
        期间一支来自Hile半岛东北部Kaernan地区的传教士为宣扬他们的诸神之王,
        带来了日神教。城邦同盟很快便放弃了他们的诸神信仰,皈依了这位诸神之王:日之神。

        征服纪115年,Vita-Avalon-IIV为了迎娶Vaski城邦主教的女儿,力排众议,于Vaski城邦与城邦同盟签订了合约,完成了婚礼。
        由于教义需要结婚双方都需要服从诸神之王,日神教也随着AvalonIIV的皈依而传入。

        AvalonIIV驾崩后,他的儿子不具有法理上的继承权,因而王位传给了他的侄子Vita-Avalon-Kuna,也是AvalonIV。
        而于此同时,日神教中产生了对于诸神之王神性的质疑,圣日教从中诞生。

        这种质疑慢慢在第二王朝蔓延开来,对诸神之王的质疑慢慢延展到对AvalonIV的质疑,
        为了稳固权力,他大规模屠杀国内的圣日教的教徒,建立了神日教。

        第二王朝和城邦同盟的对立再一次开始了,但是碍于Vaski条约的效力,双方并没有明面上的武装冲突。
        这种微妙的平衡延续了数百年,圣日教在城邦联盟广为流传,而神日教作为统治的工具,慢慢完善着。

        终于,征服纪674年,第二王朝借着诸神之王的名义焚毁了Tulaou城邦,Hile宗教战争爆发了。
        战争依旧持续了百余年,圣日教对于城邦同盟的统治也变得理所当然,腐败之风很快蔓延。

        征服纪884年,圣日教颁布新教典,里面的一条圣子税彻底激怒了信徒,城邦同盟从此分裂。年轻的主教Vaton提出了神自在论,
        并将其理论作为新的解读大量流传,圣星教也就此诞生。

        而第二王朝这一方,一场宫廷政变也让他们步入了城邦同盟的后尘,不过并没有分裂多久,Vita大帝的旁系家族的Vita-Eustance被贵族们推上王位,称为第三王朝。

        Eustance是一位贤明的君主,他施行政教分离,为神日教信徒松绑。在他继位期间,
        他统一了Greaft平原和Aliki山脉,收复了第一王朝的大部分领土,
        城邦同盟从此只有Prare平原的诸邦。

        在重新统一了Hile后,Eustance履行他对于圣星教的承诺,在Inhelong海峡附近的Tuyamen岛上修建整片大陆最大的教堂也是最大的图书馆:天之书库。

        随着天之书库的修建,城邦同盟内部争斗也渐渐和解了,整个Hile半岛的注意开始朝着天之书库凝聚。魔法纪元年,随着天之书库的竣工,第二帝国和城邦同盟共同建立了Hile帝国,长达千年之久的恩怨彻底被化解了。





    \section{泛Hile时代简介}
\part{古Hile文明}
    \chapter{Rose文明}
        \section{Rose文明的原始宗教}
            Rose文明据推测最早是一支来自西方的流民,
            有着白色的皮肤和金黄的头发,他们体格高大但并不是很喜好争斗。

            他们有着很原始的女神崇拜,信奉大母神,认为每一位孩子都
            是大母神的赐福。同时,他们也是Hile半岛上最先掌握农耕的文明,
            Hile半岛南部地区的主要食物大麦最早可以在Rose文明的遗址中找
            到踪迹,虽然这项技术不清楚具体是Rose文明自己发现还是从西方
            流民时期带来的,不过能够肯定的是,Rose文明确实就是依靠种植
            大麦在Hile半岛南部获得了主导的地位。

            大母神在Rose文明中也是代表丰收的神明,伊始之歌就是是一段关于
            麦田中降临的孩子创造万物的故事,这是Rose文明仅存的流传下来的
            传说之一,也是关于他们起源的传说。而这段传说我们可以很明显感
            受到,孩童和谷物在他们眼中具有神圣的地位,而孩子长大之后,需
            要继续诞生孩子去传递赋予在自己身上的神圣。而脱离了孩童的身份
            后,谷物便是他们人生随后所要追寻的目标。

            孩童是生命,而死亡是谷物,这是他们很古朴的死亡观念。他们不会
            安排丧葬,或者他们会将离世的人们抛入麦田,任由其腐烂以更好的
            让灵魂回归大母神。

        \section{Rose文明的宗法制度}
            Rose文明是农耕文明,因而氏族关系较为紧密,不过虽然它是母系社
            会,但是冠姓权是在孩子的父亲一方。

            在Rose文明里面,女性生下来便会作为神子而被公共抚养,她们被当作
            是大母神的神力的体现。而男性则作为一般圣子由诞下他的神女放置在
            麦田中,随后通知他的生父去领取并养育。

            圣子被领入家中后,父亲需要为举行他转灵仪式作为自己成年的象征。
            转灵仪式后,圣子获得了父亲的灵,而父亲则失去了他的灵,转而以名
            作为自己余生的身份。

            而神子出生后会被神女直接带入母神殿,在那里养育她们并传授她们成
            为神女的知识,等到丰收节后与圣子交配诞下孩童后成为真正的神女。

            成年后的圣子失去了他的灵,但是这不意味着他就失去了他的神圣,如果
            他向母神殿贡献谷物,他就再度获得了与神女交合的机会,若能诞下新的
            孩童,那便是母神收到了他的贡献,并将谷物中的神圣以孩童的形式再度
            降临于世间,而这时,若又是一位圣子,他便会由神女赋予新的灵。

            Rose的圣子作为大母神的赐下的神圣,不需要参与生产劳动,可父亲并不
            能一直只关照着圣子,所以早期他们会将圣子托付给母神殿照料,慢慢
            地,母神殿成为了最早的学校,圣子和神子一起由神女照料学习。
            \newpage
                \begin{figure}[!h]
                    \centering
                    \includegraphics[scale=0.4]{mag.png}
                    \caption{启生长廊和母神殿}
                \end{figure}
        \section{Rose文明的政治制度}
            从Rose文明的宗法制度我们可以很明显地感受到,母神殿是一个非常重要的建筑,
            事实上,确实是Rose文明部落的中枢。母神殿与其说是建筑,不如说是一片区域,
            或者一个机构。

            生育和生存是文明发展的关键,而对于母神殿,它掌管了整个文明的生育,利用生育
            要求男性上交食物以保证生存的需要。所以母神殿是整个文明的核心和中枢,在Rose
            文明中,任何重大的行动都不可能绕过母神殿去完成。

            主管母神殿的是亚母神,母神殿的管理人员则是圣女,母神殿同时还承担着教育神子
            和圣子的功能,可以说女性控制了全部的权力,这是Rose文明母系社会的典型体现。
            在部落中,男性主要负责进行农业耕作,农闲时则会负责修建房屋、城墙和宫殿等,
            有时也会出去狩猎。

            Rose文明有着很原始的公社制度,也有着原始城邦的雏形,城市围绕着农田和母神殿建立,
            虽然母神殿不会轻易容许男性的进入,但也并非是私人的领域,农田、耕牛和农具等也是
            整个部落的共同财产。

        \section{Rose文明的文化传统}
            Rose文明有着稳定的食物来源,并不喜欢争斗。对于大母神的信仰是他们的文化核心,
            而大母神是人类和自然之母,人和自然并非分离开的,而是互为兄弟的存在。所以他们
            在祭祀大母神的时候也会同样对自然表达感谢。
            
            人们收获后会举办隆重的祭典以表达对于自然的感谢,这里常常有一个误解,
            收获祭是表达对自然的感谢,而非面向大母神的祭祀。真正属于是对大母神祭祀行为是
            如伊始之歌所描绘的,将孩童降生于麦田以及转灵仪式,这二者才是对于大母神真正的祭祀。
            不过这两种仪式实施过于频繁,因而会简化流程,表现出来确实不如收获祭隆重和盛大。

            他们的称谓也别有讲究,神子会被其母赋予一个新的灵,而神子成为神女后并不会变换自己的灵,
            可以说这个出生便拥有的灵是她这一生的称谓。而对于圣子,出生后会继承父亲的灵,等自己成为父亲后
            会将灵转让给自己的圣子,自己则需要另寻一个名作为自称。若后续他再次获得一名圣子,与他交配的神女
            会将他的名转换为灵,重新举行转灵仪式。

            刑罚制度也较为温顺,多为驱逐,但是对于一些比较严重的罪犯,需要通过
            母神殿的审判,绑在麦田之中公示于众数日,随后便会深挖一个坑洞,将其活埋。
            根据考古能在几处麦田遗址下方发现诸多被捆绑的骸骨,可以推测行刑的地方就在麦田中央。

        \section{Rose文明的消亡}
            虽然真正灭亡的原因不可考究,不过根据推测,最有可能的是源于外族入侵。Rose文明不好争斗,
            他们即使拥有冶炼青铜的能力,可是他们的青铜制品主要是农具,其余只有母神殿中用于仪式的长枪和青铜法杖。
            那些法杖的主要也是用于增加土壤的肥力。

            圣子年幼时缺乏军事训练,男性也很少外出狩猎,缺乏充足的战斗技巧,这很难在外族的入侵下存活。
            根据附加地区的资料可以了解到,此时正值北民南下的时候,而且周围地区也有Rose文明部落
            的人流亡的记载。

            对于文明的消亡,我们很惋惜,但是确实这是无可奈何的事情,Rose文明有着优秀的文化传统,但是
            缺乏军事的保护,最终消失在历史的长河中,这是必然也是无法挽回的事实。
            不过Rose文明的文化也随着他们的流亡而传遍整个古Hile地区,成为古Hile文化中璀璨的一员,这是毋庸置疑的。
            \newpage
            \begin{figure}[!h]
                \centering
                \includegraphics[scale=0.4]{ruinofRose1.png}
                \caption{启生长廊废墟}
            \end{figure}




    \chapter{英雄时代}
        \section{山之子}
            在Yataki这片地区,有一户人家生了一个男婴,可是男婴生来怪异,不哭不闹,手上还多出来一根手指。
            他的家人很害怕,便将他那多出来的一根手指砍下,焚烧为灰烬,可是没过几天,手指便从伤口中重新长了出来。

            不同于寻常的男孩随后被他的家人抛弃了,他被丢弃在一条小溪旁的树下,任其自生自灭。
            可是,他是山之子,山神寻觅到那个男孩后就让他的使徒,山人将其带去了深山之中,而深山,对于Yataki这里的人来说,是不可触碰的禁忌。

            Yataki人居住在山脚之下,依赖着大山而生活,可是大山并不是甘于付出的,他会派出他的使徒,山人来到村庄,抢走他们的食物。
            身强力壮的山人作为神的使徒拥有非凡的力量,他们能掷出带有烈焰的长矛,也能从天上降下山神的怒火。Yataki人无法与他们抗衡,只能定期在山口的那座祭坛供奉食物。

            很多年过去了,男孩的家人们都仿佛忘了他,重新养育了一位新的孩子,这个行为本身并没有什么问题,可是这引起了男孩的嫉妒。

            据说,男孩用山的伟力召唤了他父母的灵魂,可是灵魂中已然没有了他的踪迹,有的只是另一个幼小的灵魂。
            那砍断手指的伤痛,嫉妒的心灵和被忘却的孤寂不断地灼烧着他自己的灵魂,最终,为了能让他的心灵安宁,他决定结束这一切。
            \emph{
            \\山之神啊,您为何要带我来这?
            \\是为了展现您的威严?
            \\亦或是为了让我的心灵烦躁?
            \\为什么我会被抛弃?
            \\这是您的旨意?
            \\还是我的罪过?
            \\山之神说:孩子,你是我的孩子。
            \\那我便不是他们的孩子。
            \\请借给我力量。
            \\向伤害我的人复仇。
            }

            随后,男孩便带着山之神的使徒来到了Yataki,将抛弃他的父母双手砍下,并随着尸体一起丢入小溪中。
            鲜红的血液随着流水,染红了整个Yataki。
            \emph{
            \\山之神啊,我们祈求您。
            \\伤害神子之人已经得到了应有的惩罚。
            \\为何还要给继续惩罚我们?
            \\是我们奉献的不够?
            \\若是如此。
            \\我们愿臣服于您。
            \\永世歌颂您的名。
            \\永世照顾您的神之子。
            \\永世维护您的一切。
            \\请您结束对于我们的惩罚吧。
            }

            神之怒被平息了,男孩带着他们的诉求回到了大山,接着更多的山人从大山中出来。
            这次,他们不是来索取和破坏的,他们带领着村民们在那片土地上建立了Yataki城邦,并将山之子和山神的名声传播。
            
            山神之国被山之子领导着,在神的指明下,永远被光辉的大山所守护着,而那座山,名为Yataki。
        \section{四神创世}
            最早,是泥神带领他的子民来到了这片地区,可慵懒的泥人只愿躺在大地上,久而久之,便永远粘在了地面上。
            泥神不能眼看着他的国就这样被大地夺走,便召唤了他的妻子木神。木神带领着她的子民来到了这片土地,一棵棵树人在泥土上安家。
            
            可是跟着木神到来的还有她的宿敌火神,这是一位脾气暴躁的女神,她喜欢吞噬树人以创造自己的子民。
            不过木神并不能就这么眼看着自己的子民被火神欺负而奈何,她找到了自己的情夫水神。
            
            水神是火神的丈夫,他有着压制火神的力量,在得知了木神被火神欺负后,立刻带领自己的子民同火神展开了一场战争。
            一场大水浇灭了熊熊燃烧的烈焰,留下的是枯白的树枝横亘在地面上,混杂着粘稠的泥沙。泥神不忍他的国成为那副惨样,
            于是召唤他的子民在战场中立起了一道高墙,阻碍了混沌的大水和残暴的烈焰。

            待战争结束,一切趋于平静,水神和火神知道他们闯下了大祸,主动向泥神寻求原谅,于是就让他们的子民化为了流动的火焰,
            进入了那包裹树枝的泥沙,生命就此诞生了。

            这四神共同的造物称之为人类,而泥神立起高墙之地,称之为Holande,意为护佑之地。
            水神和火神则在战争之时会展现他们的神威,用那流动的火焰召唤他们的子民为人类而战。
            木神则是生育与建造的守护神,她撑起了人类的身体,并让她的子民作为人类的栖所,她教会人类耕作,与她的子民一起生存和繁衍。
        \section{窃剑}
            人类曾是神的一员,可因为掌握了铸造刀剑的能力而遭到诸神的驱逐,
            为了防止人类的后代报复,他们夺走了一切关于制造刀剑的知识和能力,直到一个名叫Alas的小偷,
            他偷偷从诸神的手中偷走了那些诸神利用人类技术铸造的一把剑。然而诸神半路用长枪刺穿了他的胸膛,
            但是幸运的是他偷走的那把剑从空中跌落,狠狠地插在了地上,无法立足于大地之上的诸神无论如何也无法拔出,
            从此,人类便获得了铸造刀剑的能力,而为了纪念Alas,就用他的名字命名了那插入了剑的土地。
        \section{Iutopg之战}
            害怕人类获得技术的诸神不安于获得剑的人类,于是就蛊惑了边陲的野人Iutopg,让他们去进攻Alas,也就爆发了Iutopg之战。
            战争虽然是Iutopg进攻Alas,但是很快便被反攻了回去,不过Iutopg的城墙也很坚固,Alas无法攻破,随后Alas就撤军,
            Iutopg以为没事了,就打开了城门,Alas此时并没有进攻,而是让人假装旅者偷偷地进入了城堡,随后再发动进攻,潜入城堡的旅者偷偷地打开了城门,
            Iutopg被攻陷,Alas除掉了诸神的使者,取得了Iutopg之战的胜利。
    \chapter{希尔化时代}
        \section{八十四国}
            为了方便统治,Alas将他的领土区分为八十四个国,这八十四个国互相独立,只听令于国主议会。
        \section{诸神的和解}
            信仰自由的标志,基于Alas的开国历,他们是和诸神敌对的,但是随着信仰的愈发复杂,他们只能宣布同诸神和解,并且允许公开祭拜诸神,在之前他们是
            只允许纪念他们的先祖人类之神的。
        \section{互国行道}
            修路
        \section{Osla花园}
            一个极富盛名的学校
        \section{公共大浴场}
            Alas帝国中的第一个公共服务设施。
        \section{三层四部}
            Alas里面城邦建造的基础体系,神殿建筑为中心,公民居住在第二层,奴隶、军队和异邦人居住在第三层。
            四个部分是私人建筑、公共建筑、仓储和大广场。原则是大广场为中心连接通往各层的主干道,公共建筑环绕大广场建造,私人建筑其次,最后是仓储部分。
        \section{Daryh山会战}
            Alas与Aliki山脉的诸邦的战争,这场以少胜多的战争后Aliki山脉选择尊重Alas的统治者。
    \chapter{黑暗时代}

    \chapter{城邦联盟时代}




\part{后Hile时代}
    \chapter{泛Hile化时代}

    \chapter{Prvalaxi帝国的分裂}

    \chapter{战国时代}

    \chapter{第一王朝}

    \chapter{Vita大帝北征}

    \chapter{农民起义与诸侯叛乱}

    \chapter{第二王朝}

    \chapter{城邦同盟}


\part{宗教战争}

        第二王朝和城邦同盟之间的斗争持续了有百年之久,天灾和人祸肆虐在这片土地上,人民苦不堪言,
        他们迫切地需要得到救赎。也就是于此时,一支来自Hile半岛东北部Kaernan地区的传教士为宣扬他们的诸神之王,
        带来了日神教。城邦同盟很快便放弃了他们的诸神信仰,皈依了这位诸神之王:日之神。

        征服纪115年,Vita-Avalon-IIV为了迎娶Vaski城邦主教的女儿,背叛了祖训,放弃了恢复第一王朝的伟业,宣布与城邦同盟和解。
        那一年,百余年的战争结束了,双方于Vaski城邦签订了合约,完成了婚礼。同时,按照日神教的教义,AvalonIIV需要臣服于诸神之王才能与收到诸神庇护的女子结婚。
        虽然第二王朝有诸多反对,但是AvalonIIV利用祖父的威望力排众议,执意完成了婚礼,
        日神教得以传入。

        AvalonIIV驾崩后,他的儿子不具有法理上的继承权,因而王位传给了他的侄子Vita-Avalon-Kuna,也是AvalonIV。AvalonIV虽然并不是诸神信仰的信徒,但是他也并不愿意归顺诸神之王。
        而于此同时,日神教中产生了关于世界善恶本质的大讨论,继而引发了对于诸神之王的质疑,圣日教从中诞生。

        这种质疑慢慢在第二王朝蔓延开来,对诸神之王的质疑慢慢延展到对AvalonIV的质疑,为了稳固权力,AvalonIV曲解日神教的教义,
        大规模屠杀国内的圣日教的教徒,随后在日神教的基础上建立了神日教作为自己统治的手段。

        第二王朝和城邦同盟的对立再一次开始了,只不过这一次演化为了神日教和圣日教的对立,但是碍于Vaski条约的效力,双方并没有明面上的武装冲突。
        这种微妙的平衡延续了数百年,圣日教在城邦联盟广为流传,而神日教作为统治的工具,经历数代的改进,已经成为了一个高度严苛且完善的宗教体系。

        终于,征服纪674年,第二王朝借着诸神之王的名义焚毁了Tulaou城邦,Hile宗教战争爆发了。城邦同盟很快便在圣日教的领导下组建了反击的军队,Hile地区再度燃烧在战火之中。

        战争依旧持续了百余年,可也正是如此,圣日教对于城邦同盟的统治也愈发得理所当然,而圣日教是一个世俗化的教派,腐败之风很快蔓延,
        地区主教借着战争贪敛了大量财富,为本就承受战火的人民再添加了一份税赋。常年的战争同时使得第二王朝内部十分不稳定,双方都处于摇摇欲坠的状态。

        征服纪884年,圣日教颁布新教典,里面的一条圣子税彻底激怒了信徒,城邦同盟从此分裂。年轻的主教Vaton提出了神自在论,并将其理论作为新的解读大量流传,圣星教也就此诞生。

        而第二王朝这一方,一场宫廷政变也让他们步入了城邦同盟的后尘,不过并没有分裂多久,Vita大帝的旁系家族的Vita-Eustance被贵族们推上王位,称为第三王朝。

        Eustance是一位贤明的君主,他解除了政教合一的形式,为神日教信徒松绑。在他继位期间,
        他统一了Greaft平原和Aliki山脉,收复了第一王朝的大部分领土,
        城邦同盟从此只有Prare平原的诸邦。

        在重新统一了Hile后,Eustance履行他对于圣星教的承诺,在Inhelong海峡附近的Tuyamen岛上修建整片大陆最大的教堂也是最大的图书馆:天之书库。

        随着天之书库的修建,城邦同盟内部争斗也渐渐和解了,整个Hile半岛的注意开始朝着天之书库凝聚。魔法纪元年,随着天之书库的竣工,第二帝国和城邦同盟共同建立了Hile帝国,长达千年之久的恩怨彻底被化解了。


    \chapter{日神教}

    \chapter{Vaski条约}

    \chapter{神启之战}

    \chapter{圣灵的召唤}

    \chapter{第三王朝}

    \chapter{天之书库}

    \chapter{国教}

    \chapter{Hile帝国}

\part{Hile民族}
    \chapter{新世界}
    
    \chapter{地理大发现}



\end{document}