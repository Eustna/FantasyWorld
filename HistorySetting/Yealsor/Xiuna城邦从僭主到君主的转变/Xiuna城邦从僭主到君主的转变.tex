\documentclass[12pt, a4paper]{ctexart}
\usepackage{amsmath, amsthm, amssymb, appendix, bm, graphicx, hyperref, mathrsfs}
\title{Xiuna城邦从僭主到君主的转变}
\author{}
\date{}

\begin{document}
\maketitle
\newpage
\begin{abstract}
    本文简述了Xiuna城邦早期从僭主政治转向君主制的过程,
    详细探讨了Xiuna城邦政治变革的历史,
    阐明了Xiuna城邦的君主制及后续Queto地区乃至Yealsor文明的君主制的发展基础。
    \par\textbf{关键词:}Xiuan城邦,君主制,政治
\end{abstract}

\section{僭主政治的源起与发展}
    \subsection{早期城邦时代的政治格局}
        在Queto地区早期城邦逐步形成的时候,
        城邦这一概念比较模糊,
        普遍还是以当地的氏族贵族为主的长老会议制。
        随着社会进入奴隶制社会,
        大批奴隶主在这个时期形成,
        为了能够有效地管理奴隶,
        许多奴隶主和氏族贵族聚在了一起,
        形成了早期城邦。

        这些早期城邦基本都是采用了元老院制度,
        标志着奴隶主和氏族贵族的共和。
        而事实上,
        年轻的奴隶主们一直试图去谋求更多的权力,
        但这意味着本地的氏族贵族需要让出自己的权力,
        他们不可能会答应。

        即使是在元老院把持下的共和制下,
        奴隶主和氏族贵族也并不是那么地共和,
        时常有因为奴隶售卖和奴隶庇护等问题产生过不少争端。
        最为著名的就是Zipor城邦中的偷奴事件,
        当地的贵族看上了某奴隶主的家奴,
        而这位家奴并不作为商品出售,
        因此贵族采取了绑架的手段强行霸占了那位奴隶。
        这种行为是赤裸裸的抢劫行为,
        当时引起了Zipor城邦的一片哗然,
        但是奴隶主们并没有任何办法去对待这些贵族们,
        他们的特权就是允许他们这样飞扬跋扈。
        所以Zipor城邦的奴隶主们没有试图继续与贵族们对抗,
        而是选择试图融入他们,
        成为贵族的一员去享受这种特权。

        Zipor城邦的情况是普遍的,
        不过也有许多城邦与其相反,
        当地的氏族贵族并没有多少权力,
        城邦被奴隶主和自由民所主导,
        再随着后续的奴隶主之间的兼并,
        有些颇有权望的奴隶主则把持了元老院,
        成为了城邦的实际统治者,
        这也就是僭主。
        但僭主政治是短暂的,
        每当僭主死亡时,
        权力就会被重新分配,
        虽然其亲族会继承他的奴隶,
        但是城邦的贵族和其他奴隶主则会快速地利用元老院去分割他的权力和遗产。
        僭主很少有能将自己权力顺利交接的,
        即使有人能够继承,
        但也没出现过继续接替的情况,
        往往此时已经被元老院所架空,
        能够安稳结束此生都将会成为一种奢望。

        所以僭主们虽然拥有过人的才能,
        但是往往在他们死后,
        他们亲族的命运会十分悲惨,
        也少有人歌颂这些僭主们的成就,
        对于其评价,
        历史只记录了独裁者、忤逆神明之人和暴君等词汇。
        这使得后世的奴隶主们越来越害怕去成为僭主,
        只能屈服于贵族和元老院之下。
    \subsection{贵族元老院下的僭主}
        元老院制度在城邦早期得以确立,
        起初确实有奴隶主们的一部分,
        但随着贵族不断利用自己特权大量收购奴隶,
        使得他们和奴隶主具有了相同甚至更多的财富。
        久而久之,
        奴隶主们成为了在外掠夺奴隶随便送到城邦售卖给贵族的抓奴工人,
        他们不仅要承担战争的损失,
        同时也因为长期的掠奴行为使得他们远离城邦,
        远离了政治,
        使得贵族们基本完全掌握了元老院,
        没有了他们的容身之所。

        奴隶主的衰败是历史的必然,
        因此颇有远见的一些奴隶主们则是放弃去试图向贵族索求更多的权力,
        而是通过支持贵族们扩大特权以换取自己能够成为贵族的一员。
        长此以往,
        奴隶主和贵族们便没了区别,
        而这一时期的元老院制度也自然变成了贵族元老院,
        这种贵族元老院也就是广泛意义上的元老院制度,
        也是其后代表共和体制与君主和民主体制相抗衡的基本政治制度。

        在贵族元老院下,
        奴隶主离开了,
        但取而代之的是各大家族之间的斗争。
        他们都试图去独占特权,
        以掌握更多城邦的资源,
        有些成功的则是通过军事政变成为了僭主,
        但很可惜,
        他们的结局和那些奴隶主的僭主一样,
        在他们死后,
        领兵权被收回,
        家族被流放或者处死。
        而且他们相较而言比奴隶主的僭主更为悲惨,
        因为家族的命运在他们成为僭主的那刻便被注定,
        领兵权是由城邦所赋予的,
        并没有任何继承的合法性,
        所以亲族在其死后没有任何可以依靠的事物。
        为了保全自我,
        家族成员往往众叛轻离,
        这导致哪怕他们还活着,
        也无法得到自己应有的荣耀和安宁。
\section{Xiuna城邦的僭主}
    \subsection{Xiuna城邦建立后的动荡}
        Limdis文明的君主制以Xiuna城邦的君主制为代表,
        Xiuna是Euro地区的一座城邦,
        它兴起于前LC63年,
        坐落于Garlum沼泽,
        以手工业和农业为基础,
        大量吸收了周围部落人口,
        是当时较为繁荣的城邦之一。

        随着人口的增加,
        Xiuna城邦内部的矛盾不断激增,
        最为代表的是自由民和公民之间的矛盾。
        繁荣的经济吸引了大量的自由民的加入,
        他们农忙时在城外耕作,
        农闲时则进入城邦从事手工业生产。
        这些自由民居住在城邦内脏乱的地方,
        他们大多数人并没有固定的居所,
        被当地人视为流贼百般提防。

        前LC38年中冬季,
        饥荒蔓延,
        公民尚不能果腹,
        更不用说被视为流贼的自由民了,
        他们为了生存,
        只能通过偷窃和乞讨的方式获得一点食物。
        可越来越多的偷窃事件发生,
        不仅没有解决自由民的困境,
        反而让那些清白的人背上了莫须有的偏见甚至罪名。

        前LC38年中冬季7月,
        一些公民兵出于一己私利,
        借以抓窃贼的名义冲入贫民区,
        对当地的自由民肆意的掠夺和拘捕,
        夺走了那些他们本就所剩无几的财产。
        这一行为引起了自由民的不安,
        随后便激起了不满,
        暴动随之发生。

        元老院指派Farane家族的\emph{Ielous}进行镇压,
        公民兵很快封锁了街区,
        一天之内便平息了暴动,
        拘捕并当场处决了大量的暴乱人群。
        但是这只是表面上的,
        待夜晚降临,
        暗杀事件四处发生,
        大量自由民从街巷中涌出,
        夺取了位于城郊的武器库。

        暴动演变为了一场战争,
        \emph{Ielous}面对的不再是手无寸铁的平民,
        而是具有武装且极度愤怒的士兵,
        为了避免冲突的扩大,
        \emph{Ielous}私下和暴民们进行了谈判,
        当即处决了那些肆意闯入贫民区的公民兵和一些负责镇压的将领。
        \emph{Ielous}先斩后奏的行为让元老院很不满,
        可就在元老贵族们对\emph{Ielous}进行声讨的时候,
        \emph{Ielous}统御的公民兵和收编的自由民闯入并强行解散了元老院。
    \subsection{僭主的改革}
        \emph{Ielous}的行为和历史上其他僭主差不多,
        都是依靠军事手段解散了元老院,
        通过对元老院进行封锁,
        限制了元老院发挥其作用。
        不过在架空了元老院后的行为才是\emph{Ielous}真正区别于其他僭主的关键所在,
        \emph{Ielous}为了确保武力的稳定性,
        他组建了一支常备军,
        区别于Xiuna城邦的公民兵,
        常备军的构成人员主要为自由民和奴隶。

        这支常备军也可以叫做从兵(Dela),
        这是一种在Xiuna城邦特有的军事制度,
        从兵由城邦统一征募,
        随后以1:1的比例分派给城邦的公民兵,
        他们作为侍从暂住于公民兵的家中,
        并在非训练期间为宿主提供劳动。

        他们作战以及训练用的武器装备由他们的宿主提供,
        一般都是直接使用公民兵自己的武器,
        但之后由于许多公民兵通过制备劣质武器去骗取城邦的补偿,
        改为了由城邦统一提供武器装备。
        这不仅促使了Xiuna城邦武备的标准化和统一化,
        也保证了武力的集中,
        因为之后不仅仅是从兵的武器装备由城邦提供,
        公民兵也选择使用城邦能提供的武器。
        
        武库官(Wolunaama)也就在这一时期出现了,
        原本这只是一个负责看管武器库的小官员,
        但随着从兵武器的改革,
        武器库成了重要的军事机构,
        而武库官也成为了很重要的军事官。
        也正因如此,
        武库官在之后的葬仪之乱中发挥了关键性的作用。

        有了武力的保证后,
        \emph{Ielous}进一步改革了元老院,
        他在东城区的洼地里修建了广厅元老院,
        不同于西边土丘上的元老院,
        广厅元老院没有威严幽暗的氛围,
        而是作为一个开放的广场敞开着。
        广厅元老院吸纳了大量的公民和自由民,
        元老院的成员不再局限于贵族。

        很快广厅元老院成为了城邦中最热闹的地方,
        大家聚集在那里商议事情、观看演出以及参与审判等,
        但这种自由开敞的氛围并没有持续多久,
        广厅元老院修建好后的一年之内的下冬季,
        \emph{Ielous}的身体急速恶化,
        城邦内暗流涌动,
        许多青年贵族正在策划一场巨大的阴谋。
    \subsection{葬仪之乱}
        前LC30年早夏季,
        \emph{Ielous}病逝,
        这位伟大的僭主留在了历史之中。
        广厅元老院中一片哀悼,
        大家一致决定为其安排一场隆重的葬礼,
        将其葬在Xipinous神殿之中以希望他能以伟大之人(Abana)的身份回归于群星之中。
        而于此同时,
        土丘上幽暗的元老院中青年贵族们并没有为此感到悲伤,
        他们密谋阻止\emph{Ielous}的安眠,
        以表示群星对于\emph{Ielous}的拒斥,
        也希望用一场骚乱来表示主神Xipinous对于篡位者\emph{Ielous}的惩罚。

        不过他们的计划并非是密不透风的,
        一些广厅元老院的元老议员们早已注意到了这些青年贵族的异常举动,
        他们知道\emph{Ielous}死后将会引来一波动荡,
        这是历代僭主死后都会发生的事情,
        只是元老议员们并不清楚那些青年贵族具体想要做什么,
        但他们也不会坐以待毙,
        也在暗自准备好保护自己的力量。

        葬仪当天的清早,
        \emph{Ielous}的棺椁正被抬上渡天船时,
        通往神殿的道路被公民兵封锁,
        他们阻碍并试图抢夺\emph{Ielous}的棺椁,
        但被护送的守卫击退了。

        击退了来抢夺棺椁的暴徒后,
        元老议员很快得到了政变的消息,
        他们带着自己的侍从来一同护送棺椁。
        但此时宏伟大道已经被贵族们的士兵封锁,
        毫无任何商量和调节的余地,
        在几位元老议员倒在了乱箭之下,
        双方最终兵戎相见,
        爆发了冲突。

        元老议员的侍卫们终究不敌训练有素的士兵,
        很快就疲乏下来。
        为了保全力量,
        他们最终选择了撤退,
        打开了装殓的棺椁,
        背着\emph{Ielous}的尸体快马离开了Xiuna城邦。
        当天城邦被青年贵族们控制,
        大批从兵被捕,
        自由民们岌岌可危。

        但就在这危机的关头,
        武库官选择站在了自由民这边,
        在公民兵们大量搜捕从兵和自由民以及反对的公民兵时,
        他们选择开放了武器库,
        庇护并武装了这些人,
        使得他们有能力保护自己所居住的城区。

        虽然如此,
        但他们此时并没有反抗的合法性,
        如果等元老院不久后在公民们面前宣布了他们的回归后,
        他们都将被以叛国之罪处以极刑。
        因此,
        他们决定主动出击,
        趁着夜色主动进攻了城门并迎接了\emph{Ielous}的尸体归来。

        葬仪的第二天,
        双方僵持着,
        一方持有着\emph{Ielous}的尸体,
        而另一方则持有着\emph{Ielous}的棺椁。
        这时,
        \emph{Ielous}之子\emph{Magee}冒着被射杀的风险,
        向在场的所有人进行了一场演说,
        强调了自己为父亲送行的合法性以及父亲统治下的功绩,
        直言不讳地指出了青年贵族们内心的邪恶想法,
        使得那些阻碍的人羞愧难当。
        士兵们在围观者的呼喊声中选择了放下武器,
        只留下了那些愤怒的青年贵族无能的威胁。

        葬仪继续进行,
        这场骚乱也就此结束。
        后续那些参与政变的贵族全部被处以了极刑,
        但广厅元老院中依然留有大量贵族残余,
        阻碍了审判工作的进行。
        因此\emph{Magee}在葬仪结束后不久便立刻解散了元老院和广厅元老院,
        取而代之得设立了主政团与民政团,
        完成了审判工作后不久就废除了元老院制度,
        \emph{Magee}在之后的城邦管理中,
        基于主政团与民政团建立了主政院和民政院,
        两院统称为政院制度,
        是Xiuna城邦君主制的正式开端,
        也是从僭主到君主的根本性改变。

        在Xiuna城邦甚至整个Queto地区里,
        出现过许多类似\emph{Ielous}的独裁者,
        但是他们的统治往往伴随着他们的死亡而结束。
        而Xiuna的两院制度让僭主政治转变为了以官僚政治为基础的君主制政体,
        确保了统治者能够稳定地将权力传到继任者手中,
        这是君主制发展的基础。
\section{附加说明}
    \subsection{公民和自由民关系在不同时期的变化}
        前LC38年因为饥荒,
        公民和自由民的关系跌入冰点,
        但这不意味着双方是一直在仇视着对方。
        在\emph{Ielous}遣散了元老院后,
        采取的从兵制度在很大程度上缓和了双方的矛盾。

        对于一般的公民和自由民而言,
        除了公民需要为城邦提供兵役服务,
        而自由民有稍高的税收外,
        双方并没有什么明显的区别。
        这也意味着二者没有根本的利益冲突,
        只是自由民的存在让公民们产生了不安,
        而城区的分离更加剧了这种疏离感,
        以至于前LC38年的那场饥荒将这种不安给展现了出来。

        但从兵制让自由民寄宿在公民兵的家中,
        起初是引起了大量公民的抵制,
        但为了避免惨剧再次发生,
        需要安抚自由民的情绪,
        同时城邦也会为公民兵家庭提供补贴和足够镇压从兵的武器装备,
        这才缓和了公民们抵制的情绪。

        起初公民们是准备了一堆措施来对付这些他们认为素质低下的自由的奴隶,
        但随着后续的相处让他们发现自由民多数品行良好,
        也勤于工作,
        因此他们放下了防备的心理接受了他们,
        更有甚者某些家庭将寄宿的从兵当作自己的儿子对待。

        虽然没有严格明文禁止公民和自由民通婚,
        但长期以来不能和自由民结婚是Xiuna城邦公民的共识,
        可随着从兵制的使用,
        大量自由民融入了公民的社会之中,
        有些家庭破例让自己的小女儿和寄宿的从兵结婚,
        这打破了长期以来封闭的社会阶层,
        虽然后续广厅元老院因为古训传统的缘故,
        他们并没有获得公民身份,
        而他们的子嗣也无法获得公民身份。

        但无论如何,
        自由民们和公民们有些存在了亲族关系,
        这时候双方的矛盾已经几乎消失了。

        由于从兵开创了先河,
        自由民得以进一步渗透进入公民们的社会生活之中,
        只有顽固的古董们依然念叨着不能通婚,
        可事实上并没有多少人还在意。
        在\emph{Ielous}执政的长达8年的时间中,
        创建了无数的新家庭,
        自由民和公民在各个领域互相融合,
        早已忘记了数年前的冲突。
    \subsection{葬仪之乱的一些解释}
        虽然在葬仪之乱时候公民和自由民已经不再有很强烈的矛盾冲突,
        但贵族依然不愿意自由民能够参与城邦生活之中,
        贵族们的统治基于血缘和传统,
        这使得他们能够在公民中享有威望,
        但自由民并不会轻易承认他们贵族尊贵,
        或者说贵族们也认为不值得去谋求自由民的认可。

        所以对于自由民以及广厅元老院,
        贵族们始终是一种抵触的态度。
        而他们的抵触也并非是软弱无力的,
        恰恰相反,
        他们始终控制着公民兵,
        只是碍于\emph{Ielous}他们没法很好地去表现自己的抵触罢了。

        可一当\emph{Ielous}逝世,
        马上就爆发了葬仪之乱,
        这也是贵族们抵触能力的体现。
        当然,
        这是一个\emph{Ielous}无法解决的问题,
        因为在解散贵族元老院后,
        本就令各贵族家族不满到了极点,
        一来出于需要缓和自由民的情绪,
        另一方面\emph{Ielous}虽然解散了贵族元老院,
        但他并没有撤销各大家族的领兵权,
        在需要时,
        他们仍可以征召城邦内所统领的公民兵。

        这也就是为什么虽然葬仪之乱时公民和自由民以及融合在了一起,
        但是依然需要兵戎相见的原因,
        公民兵需要为贵族和城邦提供军事义务,
        在需要的时候听从长官的命令。

        虽然当时\emph{Ielous}没法直接强行收回各大家族的领兵权,
        但他采取了一个很聪明的做法,
        也就是从兵制度,
        这在一定程度上消解了元老贵族们对于公民兵的掌控,
        为自由民建立了武装,
        同时也加强了整个城邦的防御,
        促进了文化的传播和融合,
        推动了Xiuna城邦的繁荣,
        同时还无形间征服了周边诸多部落,
        也吸引了更多自由民前来Xiuna城邦及其周围定居。
    \subsection{武库官}
        在从兵制度之前Xiuna城邦就有修建武器库,
        但当时装备主要还是由公民们自己提供,
        不过为了应对战争到来之时,
        许多公民兵还需要置备武器的情况,
        城邦专门建造了一个武器库,
        用以存放一些工匠锻造下来的残次品,
        以确保每个公民兵都能在作战时持有武器。

        不过城邦提供的武器终究不如公民们自己置备的,
        毕竟这是战场上保命的关键,
        不可能去依赖一些发放的残次品来确保自己的安全,
        因此事实上并没有多少公民兵会需要在武器库中拿取武器,
        所以在这个时候,
        武库官是一个十分不起眼的官职,
        甚至在暴乱发生的时候,
        很多贵族都没意识到还有武器库的存在,
        直到暴民们确确实实地从武器库中拿到了些虽然残次却有足够杀伤力的武器装备。

        也就是在这时,
        武库官重新被重视起来,
        解决完骚乱后\emph{Ielous}做的第一件事就是撤换武库官,
        因为他要确保城邦的安全,
        另一方面也要确保自由民们能快速地威胁到贵族们。

        而随着从兵武器的改革,
        武器库的重要性开始体现,
        由于长期以来武库官的任命一直是\emph{Ielous}来负责,
        因此之后的武库官的选择,
        元老贵族们并没有什么机会插手干涉,
        使得武库官几乎只是\emph{Ielous}的私人官员。
        不过元老贵族们也并不是太在乎这些,
        他们依旧坚信着自己所掌握的公民兵,
        也相信他们所持有的武器装备足够精良。

        武库官名义上是城邦的官员,
        但事实上都是由\emph{Ielous}提拔,
        听从\emph{Ielous}的指使。
        当\emph{Ielous}死后,
        武库官理应是归属于城邦管理,
        但在葬仪之乱上,
        广厅元老院或者贵族元老院并没有时间来进行任免,
        这给了他们机会去庇护那些熟悉的从兵和从兵们所要保护的自由民以及一些相关的公民兵。
    \subsection{Xiuna城邦的神话的禁忌}
        在Xiuna城邦中有有一则名为安息之水传说,
        相传在上古时期,
        Xiuna城邦尚未形成的时候,
        Xiuna这片地区诞生了一位聪慧勇猛的英雄叫\emph{Dugela},
        他斩杀了盘踞沼泽的巨蛇,
        并将其献给了主神Xipinous。

        \emph{Dugela}的谄媚让当地的居民心生不满,
        嫉妒的他们最终驱逐了这位斩杀巨蛇的英雄,
        \emph{Dugela}最终客死他乡,
        含恨而终,
        可就算如此,
        嫉妒的人们依然没有放过他。
        哪怕是他儿子背着他的尸骨跪着向人们乞求,
        人们也依然不允许他的尸骨安葬在家乡,
        不允许他的灵魂回归天际。

        \emph{Dugela}的儿子在广场哭了整整七天七夜,
        他的泪水淹没了神龛,
        惊扰到了主神Xipinous,
        可祂并没有责备他,
        而是仔细地聆听他的陈述。

        主神Xipinous了解完详情后勃然大怒,
        让洪水淹没了村庄,
        让沼泽变成了湖泊,
        人们被迫攀上了玄山。
        为了忏悔自己的罪过,
        人们隆重地埋葬了英雄\emph{Dugela},
        并建造了神殿以请求\emph{Dugela}和主神Xipinous的宽恕。

        此后,
        葬礼被人们视为一件不容玷污的圣事,
        尤其是在儿子持有其父亲的尸体的时候,
        这在人们看来是会惹怒主神的行为,
        因此青年贵族们如此急切地想要去抢夺\emph{Ielous},
        原因也在于此,
        如果其子无法得到他父亲的尸骨,
        那么他便无法求得主神的帮助。
        





        




\end{document}